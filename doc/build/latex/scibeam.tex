%% Generated by Sphinx.
\def\sphinxdocclass{report}
\documentclass[letterpaper,10pt,english]{sphinxmanual}
\ifdefined\pdfpxdimen
   \let\sphinxpxdimen\pdfpxdimen\else\newdimen\sphinxpxdimen
\fi \sphinxpxdimen=.75bp\relax

\PassOptionsToPackage{warn}{textcomp}
\usepackage[utf8]{inputenc}
\ifdefined\DeclareUnicodeCharacter
 \ifdefined\DeclareUnicodeCharacterAsOptional
  \DeclareUnicodeCharacter{"00A0}{\nobreakspace}
  \DeclareUnicodeCharacter{"2500}{\sphinxunichar{2500}}
  \DeclareUnicodeCharacter{"2502}{\sphinxunichar{2502}}
  \DeclareUnicodeCharacter{"2514}{\sphinxunichar{2514}}
  \DeclareUnicodeCharacter{"251C}{\sphinxunichar{251C}}
  \DeclareUnicodeCharacter{"2572}{\textbackslash}
 \else
  \DeclareUnicodeCharacter{00A0}{\nobreakspace}
  \DeclareUnicodeCharacter{2500}{\sphinxunichar{2500}}
  \DeclareUnicodeCharacter{2502}{\sphinxunichar{2502}}
  \DeclareUnicodeCharacter{2514}{\sphinxunichar{2514}}
  \DeclareUnicodeCharacter{251C}{\sphinxunichar{251C}}
  \DeclareUnicodeCharacter{2572}{\textbackslash}
 \fi
\fi
\usepackage{cmap}
\usepackage[T1]{fontenc}
\usepackage{amsmath,amssymb,amstext}
\usepackage{babel}
\usepackage{times}
\usepackage[Bjarne]{fncychap}
\usepackage{sphinx}

\usepackage{geometry}

% Include hyperref last.
\usepackage{hyperref}
% Fix anchor placement for figures with captions.
\usepackage{hypcap}% it must be loaded after hyperref.
% Set up styles of URL: it should be placed after hyperref.
\urlstyle{same}
\addto\captionsenglish{\renewcommand{\contentsname}{GENERAL}}

\addto\captionsenglish{\renewcommand{\figurename}{Fig.}}
\addto\captionsenglish{\renewcommand{\tablename}{Table}}
\addto\captionsenglish{\renewcommand{\literalblockname}{Listing}}

\addto\captionsenglish{\renewcommand{\literalblockcontinuedname}{continued from previous page}}
\addto\captionsenglish{\renewcommand{\literalblockcontinuesname}{continues on next page}}

\addto\extrasenglish{\def\pageautorefname{page}}

\setcounter{tocdepth}{1}



\title{scibeam Documentation}
\date{Aug 27, 2018}
\release{0.1.1}
\author{Yu Lu}
\newcommand{\sphinxlogo}{\vbox{}}
\renewcommand{\releasename}{Release}
\makeindex

\begin{document}

\maketitle
\sphinxtableofcontents
\phantomsection\label{\detokenize{index::doc}}


\noindent{\hspace*{\fill}\sphinxincludegraphics[width=700\sphinxpxdimen]{{logo}.png}\hspace*{\fill}}
\sphinxhref{https://travis-ci.org/SuperYuLu/SciBeam}{\sphinxincludegraphics{{/home/yulu/Github/SciBeam/doc/build/doctrees/images/7cf4ccdd6332461ba33df6aad10a05cf0d70448f/SciBeam}.svg}}\sphinxhref{https://codecov.io/gh/SuperYuLu/SciBeam}{\sphinxincludegraphics{{/home/yulu/Github/SciBeam/doc/build/doctrees/images/672f7cfd3a931fcfcd6bb2b578a1477ec43dab56/badge}.svg}}\sphinxhref{https://scibeam.readthedocs.io/en/latest/?badge=latest}{\sphinxincludegraphics{{/home/yulu/Github/SciBeam/doc/build/doctrees/images/e850d3eccc8315e25891af5a69500e221e4e3c10/045848884e9aec5dff670df35f2731cdc6ad27a7}.svg}}\sphinxhref{https://badge.fury.io/py/scibeam}{\sphinxincludegraphics{{/home/yulu/Github/SciBeam/doc/build/doctrees/images/1e1ecccafa47fa3136c223ed1f8e784009f394f0/scibeam}.svg}}
SciBeam is a python package build on top of pandas, numpy, sicpy and matplotlib. It is  aimed for quick and easy scientific time-series data analysis and visualization in physics, optics, mechanics, and many other STEM subjects.

In the context of scientific data analysis, there are a lot of situations that people have to deal with time-series data, such as time dependent experiment(e.g. temperature measurement), dynamic processes(e.g. beam propagation, chemical reaction), system long/short term behavior(e.g. noise), etc. Quite often is that data taking and result analysis is gaped by some time and effort, which could result in complains or regrets during the data analysis,  like “I wish I took another measurement of … so than I could explain why …”. As such, the general guidline of scibeam is to bridge the gap between measurement and data analysis, so that time-series related experiment can be done in a more guided way.

The basic features of scibeam include but not limited to: beam propagation, single or multi-dimentional time depedent measurement, data file auto indexing, noise reduction, peak analysis, numerical fittings, etc.

\begin{sphinxadmonition}{note}{Note:}
scibeam doesn’t support python 2.7, make sure you have the right python version (\textgreater{}=3.4).
\end{sphinxadmonition}


\chapter{About}
\label{\detokenize{about:about}}\label{\detokenize{about::doc}}

\section{This document}
\label{\detokenize{about:this-document}}
This document is created uisng \sphinxhref{http://www.sphinx-doc.org/en/master/}{Sphinx} and \sphinxhref{https://pypi.org/project/autodoc/}{autodoc}. The general rule of the static html is configured in \sphinxhref{http://docutils.sourceforge.net/rst.html}{reStructuredText} and api document is generated by \sphinxhref{http://www.sphinx-doc.org/en/master/man/sphinx-apidoc.html}{Sphinx-apidoc}, configure in conf.py.

The general html structure looks like below:

\fvset{hllines={, ,}}%
\begin{sphinxVerbatim}[commandchars=\\\{\}]
html/
├── about.html
├── conf.html
├── genindex.html
├── index.html
├── install.html
├── \PYGZus{}modules
│   ├── index.html
│   └── scibeam
│       ├── core
│       │   ├── base.html
│       │   ├── common.html
│       │   ├── descriptor.html
│       │   ├── formatter.html
│       │   ├── gaussian.html
│       │   ├── numerical.html
│       │   ├── peak.html
│       │   ├── plot.html
│       │   ├── regexp.html
│       │   ├── tofframe.html
│       │   └── tofseries.html
│       └── tests
│           ├── test\PYGZus{}base.html
│           ├── test\PYGZus{}common.html
│           ├── test\PYGZus{}formatter.html
│           ├── test\PYGZus{}imports.html
│           ├── test\PYGZus{}regexp.html
│           └── test\PYGZus{}tofseries.html
├── modules.html
├── objects.inv
├── py\PYGZhy{}modindex.html
├── scibeam.core.html
├── scibeam.html
├── scibeam.tests.html
├── search.html
├── searchindex.js
├── setup.html
├── \PYGZus{}sources
│   ├── about.rst.txt
│   ├── conf.rst.txt
│   ├── index.rst.txt
│   ├── install.rst.txt
│   ├── modules.rst.txt
│   ├── scibeam.core.rst.txt
│   ├── scibeam.rst.txt
│   ├── scibeam.tests.rst.txt
│   ├── setup.rst.txt
│   ├── structure.rst.txt
│   └── tutorial.rst.txt
├── \PYGZus{}static
│   ├── ajax\PYGZhy{}loader.gif
│   ├── alabaster.css
│   ├── basic.css
│   ├── comment\PYGZhy{}bright.png
│   ├── comment\PYGZhy{}close.png
│   ├── comment.png
│   ├── css
│   │   ├── badge\PYGZus{}only.css
│   │   └── theme.css
│   ├── custom.css
│   ├── doctools.js
│   ├── documentation\PYGZus{}options.js
│   ├── down.png
│   ├── down\PYGZhy{}pressed.png
│   ├── file.png
│   ├── fonts
│   │   ├── fontawesome\PYGZhy{}webfont.eot
│   │   ├── fontawesome\PYGZhy{}webfont.svg
│   │   ├── fontawesome\PYGZhy{}webfont.ttf
│   │   ├── fontawesome\PYGZhy{}webfont.woff
│   │   ├── fontawesome\PYGZhy{}webfont.woff2
│   │   ├── Lato
│   │   │   ├── lato\PYGZhy{}bold.eot
│   │   │   ├── lato\PYGZhy{}bolditalic.eot
│   │   │   ├── lato\PYGZhy{}bolditalic.ttf
│   │   │   ├── lato\PYGZhy{}bolditalic.woff
│   │   │   ├── lato\PYGZhy{}bolditalic.woff2
│   │   │   ├── lato\PYGZhy{}bold.ttf
│   │   │   ├── lato\PYGZhy{}bold.woff
│   │   │   ├── lato\PYGZhy{}bold.woff2
│   │   │   ├── lato\PYGZhy{}italic.eot
│   │   │   ├── lato\PYGZhy{}italic.ttf
│   │   │   ├── lato\PYGZhy{}italic.woff
│   │   │   ├── lato\PYGZhy{}italic.woff2
│   │   │   ├── lato\PYGZhy{}regular.eot
│   │   │   ├── lato\PYGZhy{}regular.ttf
│   │   │   ├── lato\PYGZhy{}regular.woff
│   │   │   └── lato\PYGZhy{}regular.woff2
│   │   └── RobotoSlab
│   │       ├── roboto\PYGZhy{}slab\PYGZhy{}v7\PYGZhy{}bold.eot
│   │       ├── roboto\PYGZhy{}slab\PYGZhy{}v7\PYGZhy{}bold.ttf
│   │       ├── roboto\PYGZhy{}slab\PYGZhy{}v7\PYGZhy{}bold.woff
│   │       ├── roboto\PYGZhy{}slab\PYGZhy{}v7\PYGZhy{}bold.woff2
│   │       ├── roboto\PYGZhy{}slab\PYGZhy{}v7\PYGZhy{}regular.eot
│   │       ├── roboto\PYGZhy{}slab\PYGZhy{}v7\PYGZhy{}regular.ttf
│   │       ├── roboto\PYGZhy{}slab\PYGZhy{}v7\PYGZhy{}regular.woff
│   │       └── roboto\PYGZhy{}slab\PYGZhy{}v7\PYGZhy{}regular.woff2
│   ├── jquery\PYGZhy{}3.2.1.js
│   ├── jquery.js
│   ├── js
│   │   ├── modernizr.min.js
│   │   └── theme.js
│   ├── minus.png
│   ├── plus.png
│   ├── pygments.css
│   ├── searchtools.js
│   ├── underscore\PYGZhy{}1.3.1.js
│   ├── underscore.js
│   ├── up.png
│   ├── up\PYGZhy{}pressed.png
│   └── websupport.js
├── structure.html
└── tutorial.html

11 directories, 101 files
\end{sphinxVerbatim}


\section{reStructuredText Sturcture}
\label{\detokenize{about:restructuredtext-sturcture}}
The reStructuredText files are the source that these htmls are build on top of. Most of the text related .rst fils are wrote in the corresponding mark up formart, other module related .rst are build using autodoc, which automatically looks in to the doc strings in python source files.

In this project, the document style in the pyhon source files are following \sphinxhref{https://sphinxcontrib-napoleon.readthedocs.io/en/latest/example\_numpy.html}{numpy style}, which is rendered by Spnhinx externsion \sphinxhref{https://sphinxcontrib-napoleon.readthedocs.io/en/latest/index.html}{napoleon}.

The structure of .rst folder structure:

\fvset{hllines={, ,}}%
\begin{sphinxVerbatim}[commandchars=\\\{\}]
├── about.rst
├── conf.py
├── index.rst
├── install.rst
├── modules.rst
├── scibeam.core.rst
├── scibeam.rst
├── scibeam.tests.rst
├── \PYGZus{}static
├── structure.rst
└── \PYGZus{}templates

2 directories, 11 files
\end{sphinxVerbatim}


\section{Package structure}
\label{\detokenize{about:package-structure}}
The package structure of scibeam is

\fvset{hllines={, ,}}%
\begin{sphinxVerbatim}[commandchars=\\\{\}]
scibeam
├── core
│   ├── base.py
│   ├── common.py
│   ├── descriptor.py
│   ├── dictfunc.py
│   ├── formatter.py
│   ├── gaussian.py
│   ├── \PYGZus{}\PYGZus{}init\PYGZus{}\PYGZus{}.py
│   ├── numerical.py
│   ├── peak.py
│   ├── plot.py
│   ├── regexp.py
│   ├── tofframe.py
│   └── tofseries.py
├── data
│   ├── examples
│   └── test
├── \PYGZus{}\PYGZus{}init\PYGZus{}\PYGZus{}.py
├── tests
│   ├── \PYGZus{}\PYGZus{}init\PYGZus{}\PYGZus{}.py
│   ├── \PYGZus{}\PYGZus{}pycache\PYGZus{}\PYGZus{}
│   ├── test\PYGZus{}base.py
│   ├── test\PYGZus{}common.py
│   ├── test\PYGZus{}formatter.py
│   ├── test\PYGZus{}imports.py
│   ├── test\PYGZus{}regexp.py
│   └── test\PYGZus{}tofseries.py
└── util
    ├── folderstruct.py
    ├── \PYGZus{}\PYGZus{}init\PYGZus{}\PYGZus{}.py
    ├── io.py
    ├── multiframe.py
    └── pipeline.py
\end{sphinxVerbatim}

10 directories, 34 files

Where:
\begin{itemize}
\item {} 
core: main part of the pacaage

\item {} 
tests: unittests

\item {} 
util: extral add ons for the package

\item {} 
data: test data and example data files

\end{itemize}


\chapter{Install}
\label{\detokenize{install:install}}\label{\detokenize{install::doc}}
Install scibeam is easy, one can choose either install using pypi or from \sphinxhref{https://github.com/SuperYuLu/SciBeam}{source code} using python setuptools.


\section{Requirements}
\label{\detokenize{install:requirements}}
The scibeam package requires:
\begin{itemize}
\item {} 
Python(\textgreater{}= 3.4)

\item {} 
Numpy

\item {} 
Scipy

\item {} 
Pandas

\item {} 
matplotlib

\end{itemize}

\begin{sphinxadmonition}{note}{Note:}
scibeam doesn’t support python 2.7, make sure you have the right python version (\textgreater{}=3.4).
\end{sphinxadmonition}


\section{Using PyPI}
\label{\detokenize{install:using-pypi}}
Scibeam is avaliable on \sphinxhref{https://pypi.org/project/scibeam/}{PyPI}, one can install under python3 environment using:

\fvset{hllines={, ,}}%
\begin{sphinxVerbatim}[commandchars=\\\{\}]
\PYG{n}{pip} \PYG{n}{install} \PYG{n}{scibeam}
\end{sphinxVerbatim}

Scibeam can then be imported as:

\fvset{hllines={, ,}}%
\begin{sphinxVerbatim}[commandchars=\\\{\}]
\PYG{k+kn}{import} \PYG{n+nn}{scibeam}
\end{sphinxVerbatim}


\section{Using Setuptools}
\label{\detokenize{install:using-setuptools}}
To install using python setuptools, simply clone the source code:

\fvset{hllines={, ,}}%
\begin{sphinxVerbatim}[commandchars=\\\{\}]
\PYG{n}{git} \PYG{n}{clone} \PYG{n}{git}\PYG{n+nd}{@github}\PYG{o}{.}\PYG{n}{com}\PYG{p}{:}\PYG{n}{SuperYuLu}\PYG{o}{/}\PYG{n}{SciBeam}\PYG{o}{.}\PYG{n}{git}
\end{sphinxVerbatim}

Then change into the SciBeam folder:

\fvset{hllines={, ,}}%
\begin{sphinxVerbatim}[commandchars=\\\{\}]
\PYG{n}{cd} \PYG{n}{SciBeam}
\end{sphinxVerbatim}

Under SciBeam folder, install by typing:

\fvset{hllines={, ,}}%
\begin{sphinxVerbatim}[commandchars=\\\{\}]
\PYG{n}{python} \PYG{n}{setup}\PYG{o}{.}\PYG{n}{py} \PYG{n}{install}
\end{sphinxVerbatim}

scibeam package name should be then available in the python environment, to import:

\fvset{hllines={, ,}}%
\begin{sphinxVerbatim}[commandchars=\\\{\}]
\PYG{k+kn}{import} \PYG{n+nn}{scibeam}
\end{sphinxVerbatim}

or:

\fvset{hllines={, ,}}%
\begin{sphinxVerbatim}[commandchars=\\\{\}]
\PYG{k+kn}{from} \PYG{n+nn}{scibeam} \PYG{k}{import} \PYG{o}{*}
\end{sphinxVerbatim}


\chapter{How to use}
\label{\detokenize{how_to_use:how-to-use}}\label{\detokenize{how_to_use::doc}}
How to use


\chapter{scibeam.core package}
\label{\detokenize{scibeam.core:scibeam-core-package}}\label{\detokenize{scibeam.core::doc}}

\section{Submodules}
\label{\detokenize{scibeam.core:submodules}}

\section{scibeam.core.base module}
\label{\detokenize{scibeam.core:module-scibeam.core.base}}\label{\detokenize{scibeam.core:scibeam-core-base-module}}\index{scibeam.core.base (module)}
Base functions for mixin classes and module width constants
\index{\_mixin\_class (in module scibeam.core.base)}

\begin{fulllineitems}
\phantomsection\label{\detokenize{scibeam.core:scibeam.core.base._mixin_class}}\pysigline{\sphinxcode{\sphinxupquote{scibeam.core.base.}}\sphinxbfcode{\sphinxupquote{\_mixin\_class}}}
\sphinxstyleemphasis{list(str)} \textendash{} Specify allowed mixin class for method chain.
The two basic data structures are TOFSeries and TOFFrame,
current.

\end{fulllineitems}


\begin{sphinxadmonition}{note}{Note:}
TODO: Move Defaults to a seperate config.py file for easy
configuration
\end{sphinxadmonition}
\index{Defaults (class in scibeam.core.base)}

\begin{fulllineitems}
\phantomsection\label{\detokenize{scibeam.core:scibeam.core.base.Defaults}}\pysigline{\sphinxbfcode{\sphinxupquote{class }}\sphinxcode{\sphinxupquote{scibeam.core.base.}}\sphinxbfcode{\sphinxupquote{Defaults}}}
Bases: \sphinxcode{\sphinxupquote{object}}

Module level default values

Settings for global default values

\begin{sphinxadmonition}{note}{Note:}
TODO: realize these using a seperate config.py file
\end{sphinxadmonition}
\index{data\_file\_extenstion (scibeam.core.base.Defaults attribute)}

\begin{fulllineitems}
\phantomsection\label{\detokenize{scibeam.core:scibeam.core.base.Defaults.data_file_extenstion}}\pysigline{\sphinxbfcode{\sphinxupquote{data\_file\_extenstion}}\sphinxbfcode{\sphinxupquote{ = '.lvm'}}}
\end{fulllineitems}

\index{data\_file\_num\_column (scibeam.core.base.Defaults attribute)}

\begin{fulllineitems}
\phantomsection\label{\detokenize{scibeam.core:scibeam.core.base.Defaults.data_file_num_column}}\pysigline{\sphinxbfcode{\sphinxupquote{data\_file\_num\_column}}\sphinxbfcode{\sphinxupquote{ = 2}}}
\end{fulllineitems}

\index{file\_regex (scibeam.core.base.Defaults attribute)}

\begin{fulllineitems}
\phantomsection\label{\detokenize{scibeam.core:scibeam.core.base.Defaults.file_regex}}\pysigline{\sphinxbfcode{\sphinxupquote{file\_regex}}\sphinxbfcode{\sphinxupquote{ = '.*\_(\textbackslash{}\textbackslash{}d+\textbackslash{}\textbackslash{}.?\textbackslash{}\textbackslash{}d+).*.lvm\$'}}}
\end{fulllineitems}

\index{subfolder\_regex (scibeam.core.base.Defaults attribute)}

\begin{fulllineitems}
\phantomsection\label{\detokenize{scibeam.core:scibeam.core.base.Defaults.subfolder_regex}}\pysigline{\sphinxbfcode{\sphinxupquote{subfolder\_regex}}\sphinxbfcode{\sphinxupquote{ = '.*(\textbackslash{}\textbackslash{}d+\textbackslash{}\textbackslash{}.?\textbackslash{}\textbackslash{}d+).*'}}}
\end{fulllineitems}


\end{fulllineitems}



\section{scibeam.core.common module}
\label{\detokenize{scibeam.core:module-scibeam.core.common}}\label{\detokenize{scibeam.core:scibeam-core-common-module}}\index{scibeam.core.common (module)}
Common functions used across classes and modules
\index{winPathHandler() (in module scibeam.core.common)}

\begin{fulllineitems}
\phantomsection\label{\detokenize{scibeam.core:scibeam.core.common.winPathHandler}}\pysiglinewithargsret{\sphinxcode{\sphinxupquote{scibeam.core.common.}}\sphinxbfcode{\sphinxupquote{winPathHandler}}}{\emph{args}}{}
A windows path string handler

Convert windows path string variables to python/linux compatible Path
\begin{quote}\begin{description}
\item[{Parameters}] \leavevmode
\sphinxstyleliteralstrong{\sphinxupquote{args}} (\sphinxstyleliteralemphasis{\sphinxupquote{string}}) \textendash{} A single or list of strings of path

\item[{Returns}] \leavevmode
Reformated string of list of strings

\item[{Return type}] \leavevmode
string

\end{description}\end{quote}

\end{fulllineitems}

\index{loadFile() (in module scibeam.core.common)}

\begin{fulllineitems}
\phantomsection\label{\detokenize{scibeam.core:scibeam.core.common.loadFile}}\pysiglinewithargsret{\sphinxcode{\sphinxupquote{scibeam.core.common.}}\sphinxbfcode{\sphinxupquote{loadFile}}}{\emph{filename}, \emph{cols=2}, \emph{usecols=None}, \emph{skiprows=0}, \emph{kind='txt'}, \emph{sep='\textbackslash{}t'}}{}
File loader

Loading txt / lvm data files
\begin{quote}\begin{description}
\item[{Parameters}] \leavevmode\begin{itemize}
\item {} 
\sphinxstyleliteralstrong{\sphinxupquote{filename}} (\sphinxstyleliteralemphasis{\sphinxupquote{string}}) \textendash{} Filename string (including the full path to the file)

\item {} 
\sphinxstyleliteralstrong{\sphinxupquote{cols}} (\sphinxstyleliteralemphasis{\sphinxupquote{int}}) \textendash{} Total number of columns to be loaded, default 2

\item {} 
\sphinxstyleliteralstrong{\sphinxupquote{usecols}} (\sphinxstyleliteralemphasis{\sphinxupquote{int}}) \textendash{} Column to be used, if None then load all. Default None

\item {} 
\sphinxstyleliteralstrong{\sphinxupquote{skiprows}} (\sphinxstyleliteralemphasis{\sphinxupquote{int}}) \textendash{} Number of rows to skip when loading data, this is specifically designed
for the case that there is header in the file

\item {} 
\sphinxstyleliteralstrong{\sphinxupquote{kind}} (\sphinxstyleliteralemphasis{\sphinxupquote{string}}) \textendash{} File format, default ‘txt’.
Currently only works for txt-like files

\item {} 
\sphinxstyleliteralstrong{\sphinxupquote{sep}} (\sphinxstyleliteralemphasis{\sphinxupquote{string}}) \textendash{} Seperator of the data column, default ‘ ‘

\end{itemize}

\item[{Returns}] \leavevmode
data loaded as numpy ndarray, default 2D array

\item[{Return type}] \leavevmode
numpy.ndarray

\item[{Raises}] \leavevmode\begin{itemize}
\item {} 
\sphinxcode{\sphinxupquote{FileNotFoundError}} \textendash{} File not found with given filename string

\item {} 
\sphinxcode{\sphinxupquote{ValueError}} \textendash{} Data loading didn’t finish

\end{itemize}

\end{description}\end{quote}

\end{fulllineitems}



\section{scibeam.core.descriptor module}
\label{\detokenize{scibeam.core:module-scibeam.core.descriptor}}\label{\detokenize{scibeam.core:scibeam-core-descriptor-module}}\index{scibeam.core.descriptor (module)}\index{DescriptorMixin (class in scibeam.core.descriptor)}

\begin{fulllineitems}
\phantomsection\label{\detokenize{scibeam.core:scibeam.core.descriptor.DescriptorMixin}}\pysiglinewithargsret{\sphinxbfcode{\sphinxupquote{class }}\sphinxcode{\sphinxupquote{scibeam.core.descriptor.}}\sphinxbfcode{\sphinxupquote{DescriptorMixin}}}{\emph{descriptor\_cls}}{}
Bases: \sphinxcode{\sphinxupquote{object}}

Meta class for method chain mixin

This is a meta class to realize method chain in other classes
Read-only descriptor for class cross reference

\end{fulllineitems}



\section{scibeam.core.formatter module}
\label{\detokenize{scibeam.core:module-scibeam.core.formatter}}\label{\detokenize{scibeam.core:scibeam-core-formatter-module}}\index{scibeam.core.formatter (module)}\index{format\_dict() (in module scibeam.core.formatter)}

\begin{fulllineitems}
\phantomsection\label{\detokenize{scibeam.core:scibeam.core.formatter.format_dict}}\pysiglinewithargsret{\sphinxcode{\sphinxupquote{scibeam.core.formatter.}}\sphinxbfcode{\sphinxupquote{format\_dict}}}{\emph{rawdict}, \emph{alphabetical=True}, \emph{digits=2}}{}
dictionary to string format

Format dictionarys to strings as a list of key value pairs in each row ,
meant for printing, annotation on plot, etc.
\begin{quote}\begin{description}
\item[{Parameters}] \leavevmode\begin{itemize}
\item {} 
\sphinxstyleliteralstrong{\sphinxupquote{rawdict}} (\sphinxstyleliteralemphasis{\sphinxupquote{dictionary}}) \textendash{} raw input dictionary

\item {} 
\sphinxstyleliteralstrong{\sphinxupquote{alphabetialy}} (\sphinxstyleliteralemphasis{\sphinxupquote{bool}}) \textendash{} if true (default) arrange dict key alphabetical

\item {} 
\sphinxstyleliteralstrong{\sphinxupquote{digits}} (\sphinxstyleliteralemphasis{\sphinxupquote{int}}) \textendash{} number of digits to keep if the value the key is numerical

\end{itemize}

\item[{Returns}] \leavevmode
Formated string in seperate rows

\item[{Return type}] \leavevmode
string

\end{description}\end{quote}

\end{fulllineitems}



\section{scibeam.core.gaussian module}
\label{\detokenize{scibeam.core:module-scibeam.core.gaussian}}\label{\detokenize{scibeam.core:scibeam-core-gaussian-module}}\index{scibeam.core.gaussian (module)}\index{Gaussian (class in scibeam.core.gaussian)}

\begin{fulllineitems}
\phantomsection\label{\detokenize{scibeam.core:scibeam.core.gaussian.Gaussian}}\pysigline{\sphinxbfcode{\sphinxupquote{class }}\sphinxcode{\sphinxupquote{scibeam.core.gaussian.}}\sphinxbfcode{\sphinxupquote{Gaussian}}}
Bases: \sphinxcode{\sphinxupquote{object}}

Class for numerical gaussian funciton application

A collections of methods for gaussian analysis on the data, such as
single gaussian function, single gaussian 1d fitting, double gaussian,
double gaussian fitting, etc.
\index{doubleGaus() (scibeam.core.gaussian.Gaussian static method)}

\begin{fulllineitems}
\phantomsection\label{\detokenize{scibeam.core:scibeam.core.gaussian.Gaussian.doubleGaus}}\pysiglinewithargsret{\sphinxbfcode{\sphinxupquote{static }}\sphinxbfcode{\sphinxupquote{doubleGaus}}}{\emph{x}, \emph{a1}, \emph{x1}, \emph{sigma1}, \emph{a2}, \emph{x2}, \emph{sigma2}, \emph{y0=0}}{}
Gaussian function of two independent variables

Double gaussian function with offset ::
y = a1 * exp((x - x1)\textasciicircum{}2 / (2 * sigma1\textasciicircum{}2) + a2 * exp((x - x2)\textasciicircum{}2 / (2 * sigma2\textasciicircum{}2))
\begin{quote}\begin{description}
\item[{Parameters}] \leavevmode\begin{itemize}
\item {} 
\sphinxstyleliteralstrong{\sphinxupquote{x}} (\sphinxstyleliteralemphasis{\sphinxupquote{float}}) \textendash{} Input variable for the double gaussian function

\item {} 
\sphinxstyleliteralstrong{\sphinxupquote{a1}} (\sphinxstyleliteralemphasis{\sphinxupquote{float}}) \textendash{} Amplitude of the first gaussian variable peak

\item {} 
\sphinxstyleliteralstrong{\sphinxupquote{x1}} (\sphinxstyleliteralemphasis{\sphinxupquote{float}}) \textendash{} Peak center for the first variable gaussian peak

\item {} 
\sphinxstyleliteralstrong{\sphinxupquote{sigma1}} (\sphinxstyleliteralemphasis{\sphinxupquote{float}}) \textendash{} Sigma vlaues for the two gaussian peaks

\item {} 
\sphinxstyleliteralstrong{\sphinxupquote{a2}} (\sphinxstyleliteralemphasis{\sphinxupquote{float}}) \textendash{} Amplitude of the second gaussian variable peak

\item {} 
\sphinxstyleliteralstrong{\sphinxupquote{a2}} \textendash{} Amplitude of the first gaussian variable peak

\item {} 
\sphinxstyleliteralstrong{\sphinxupquote{x2}} (\sphinxstyleliteralemphasis{\sphinxupquote{float}}) \textendash{} Peak center for the first variable gaussian peak

\item {} 
\sphinxstyleliteralstrong{\sphinxupquote{sigma2}} (\sphinxstyleliteralemphasis{\sphinxupquote{float}}) \textendash{} Sigma vlaues for the two gaussian peaks

\item {} 
\sphinxstyleliteralstrong{\sphinxupquote{y0}} (\sphinxstyleliteralemphasis{\sphinxupquote{float}}) \textendash{} Y offset, optional, default y0 = 0

\end{itemize}

\item[{Returns}] \leavevmode


\item[{Return type}] \leavevmode
Numerical value of the double gaussian function

\end{description}\end{quote}

\end{fulllineitems}

\index{doubleGausFit() (scibeam.core.gaussian.Gaussian static method)}

\begin{fulllineitems}
\phantomsection\label{\detokenize{scibeam.core:scibeam.core.gaussian.Gaussian.doubleGausFit}}\pysiglinewithargsret{\sphinxbfcode{\sphinxupquote{static }}\sphinxbfcode{\sphinxupquote{doubleGausFit}}}{\emph{x}, \emph{y}, \emph{guessPara}, \emph{offset=False}}{}
Two independent variable gaussian fitting

Fit the data with a double gaussian function base on given
x, y data and initial guess parameters.

Unlike the 1D gaussian fitting function, one hase to provide
initial guess parameters to make sure optimal parameters could
be found.

The fitting method is based on  least square method, fitted
parameters and their covariance matrix is returned.
\begin{quote}\begin{description}
\item[{Parameters}] \leavevmode\begin{itemize}
\item {} 
\sphinxstyleliteralstrong{\sphinxupquote{x}} (\sphinxstyleliteralemphasis{\sphinxupquote{1D array}}) \textendash{} Input data x value

\item {} 
\sphinxstyleliteralstrong{\sphinxupquote{y}} (\sphinxstyleliteralemphasis{\sphinxupquote{1D array}}) \textendash{} Input data y value

\item {} 
\sphinxstyleliteralstrong{\sphinxupquote{guessPara}} (\sphinxstyleliteralemphasis{\sphinxupquote{array-like}}) \textendash{} Initial guess parameter list{[}a1, x1, sigma1, a2, x2, sigma2, y0{]}

\end{itemize}

\item[{Returns}] \leavevmode
\begin{itemize}
\item {} 
\sphinxstyleemphasis{array1} \textendash{} Fitted parameter array {[}a1, x1, simga1, a2, x2, simga1{]}

\item {} 
\sphinxstyleemphasis{array2} \textendash{} Cnveriance matrix of fitted parameters

\end{itemize}


\end{description}\end{quote}

\end{fulllineitems}

\index{gaus() (scibeam.core.gaussian.Gaussian static method)}

\begin{fulllineitems}
\phantomsection\label{\detokenize{scibeam.core:scibeam.core.gaussian.Gaussian.gaus}}\pysiglinewithargsret{\sphinxbfcode{\sphinxupquote{static }}\sphinxbfcode{\sphinxupquote{gaus}}}{\emph{x}, \emph{A}, \emph{x0}, \emph{sigma}, \emph{offset=0}}{}
gaussian function with or without offset

General form of a 1D gaussian function, with variable as first
parameter and other associate parameters followed. Can be used
for fitting or line plotting after fitting is done.

The function generally follow the form ::
y = A * exp(-(x - x0)\textasciicircum{}2 / (2 * sigma\textasciicircum{}2)) + offset (optional)

Handles the case with and without offset seperatelly, since for
fitting without offset at all one has to force the function to
be of not offset.
\begin{quote}\begin{description}
\item[{Parameters}] \leavevmode\begin{itemize}
\item {} 
\sphinxstyleliteralstrong{\sphinxupquote{x}} (\sphinxstyleliteralemphasis{\sphinxupquote{float}}) \textendash{} variable x in gaussian function

\item {} 
\sphinxstyleliteralstrong{\sphinxupquote{A}} (\sphinxstyleliteralemphasis{\sphinxupquote{float}}) \textendash{} Peak value

\item {} 
\sphinxstyleliteralstrong{\sphinxupquote{x0}} (\sphinxstyleliteralemphasis{\sphinxupquote{float}}) \textendash{} Center coordinates

\item {} 
\sphinxstyleliteralstrong{\sphinxupquote{sigma}} (\sphinxstyleliteralemphasis{\sphinxupquote{float}}) \textendash{} Standard deviation

\item {} 
\sphinxstyleliteralstrong{\sphinxupquote{offset}} (\sphinxstyleliteralemphasis{\sphinxupquote{float}}) \textendash{} overall offset, default 0

\end{itemize}

\end{description}\end{quote}

\end{fulllineitems}

\index{gausFit() (scibeam.core.gaussian.Gaussian static method)}

\begin{fulllineitems}
\phantomsection\label{\detokenize{scibeam.core:scibeam.core.gaussian.Gaussian.gausFit}}\pysiglinewithargsret{\sphinxbfcode{\sphinxupquote{static }}\sphinxbfcode{\sphinxupquote{gausFit}}}{\emph{x}, \emph{y}, \emph{offset=False}, \emph{plot=False}}{}
Perform gaussian fit on given data

Fit data with 1D gausian function ::
y = a * exp((x - x0)\textasciicircum{}2 / (2 * sigma)) + y0(optional)

The function generates initial guesses automatically based on
given data, the algorithm is based on scipy curve\_fit function
\begin{quote}\begin{description}
\item[{Parameters}] \leavevmode\begin{itemize}
\item {} 
\sphinxstyleliteralstrong{\sphinxupquote{x}} (\sphinxstyleliteralemphasis{\sphinxupquote{array-like}}) \textendash{} X values of the input data

\item {} 
\sphinxstyleliteralstrong{\sphinxupquote{y}} (\sphinxstyleliteralemphasis{\sphinxupquote{array-like}}) \textendash{} Y values of the input data

\item {} 
\sphinxstyleliteralstrong{\sphinxupquote{offset}} (\sphinxstyleliteralemphasis{\sphinxupquote{bool}}) \textendash{} Wether fit gaussian with offset or not
Default False

\item {} 
\sphinxstyleliteralstrong{\sphinxupquote{plot}} (\sphinxstyleliteralemphasis{\sphinxupquote{bool}}) \textendash{} Wether plot the fitting result or not
Default False

\end{itemize}

\item[{Returns}] \leavevmode
\begin{itemize}
\item {} 
\sphinxstyleemphasis{array1} \textendash{} Array of optmized best fit data {[}a, x0, sigma, y0{]}

\item {} 
\sphinxstyleemphasis{array2} \textendash{} A 4 x 4 covariant matrix of the corresponding optmized data

\end{itemize}


\item[{Raises}] \leavevmode
\sphinxcode{\sphinxupquote{RuntimeError}} \textendash{} When optimized parameters not found within max depth of iteration

\end{description}\end{quote}

\end{fulllineitems}


\end{fulllineitems}



\section{scibeam.core.numerical module}
\label{\detokenize{scibeam.core:module-scibeam.core.numerical}}\label{\detokenize{scibeam.core:scibeam-core-numerical-module}}\index{scibeam.core.numerical (module)}\index{bandPassFilter() (in module scibeam.core.numerical)}

\begin{fulllineitems}
\phantomsection\label{\detokenize{scibeam.core:scibeam.core.numerical.bandPassFilter}}\pysiglinewithargsret{\sphinxcode{\sphinxupquote{scibeam.core.numerical.}}\sphinxbfcode{\sphinxupquote{bandPassFilter}}}{\emph{data}, \emph{tStep=None}, \emph{lowFreq=0}, \emph{highFreq=10000.0}}{}
band pass filter based on fourier transform

Filter the noise in time series data with given frequency range.

The data has to be in numpy array. If only 1D array is provided, one also
needs to provide time step size. If 2D array is provided, the 0th column
will be used to calculate time step size, while the 1st column will be
treated as the signal value.
\begin{quote}\begin{description}
\item[{Parameters}] \leavevmode\begin{itemize}
\item {} 
\sphinxstyleliteralstrong{\sphinxupquote{data}} (\sphinxstyleliteralemphasis{\sphinxupquote{numpy array}}) \textendash{} The input time series data.
1d array is treated as the signal value, which requires input parameter
tStep to be not None.

\item {} 
\sphinxstyleliteralstrong{\sphinxupquote{tStep}} (\sphinxstyleliteralemphasis{\sphinxupquote{float}}) \textendash{} Time step size in seconds of the time series data.
If None (default), 0st columns in data will be treated as time and time
step size will be extracted from there

\item {} 
\sphinxstyleliteralstrong{\sphinxupquote{lowFreq}} (\sphinxstyleliteralemphasis{\sphinxupquote{float}}) \textendash{} Lower bound of the bandpass filter, default 0 Hz

\item {} 
\sphinxstyleliteralstrong{\sphinxupquote{highFreq}} (\sphinxstyleliteralemphasis{\sphinxupquote{float}}) \textendash{} Upper bound of the bandpass filter, default 1e4 Hz

\end{itemize}

\end{description}\end{quote}

\begin{sphinxadmonition}{note}{Note:}
The data has to be uniformly sampled, e.g. same time gap between each data
point, all parameters here are supposed to be SI unit.
\end{sphinxadmonition}

\end{fulllineitems}

\index{integrate() (in module scibeam.core.numerical)}

\begin{fulllineitems}
\phantomsection\label{\detokenize{scibeam.core:scibeam.core.numerical.integrate}}\pysiglinewithargsret{\sphinxcode{\sphinxupquote{scibeam.core.numerical.}}\sphinxbfcode{\sphinxupquote{integrate}}}{\emph{x=None}, \emph{y=None}, \emph{kind='numerical'}, \emph{func=None}, \emph{args=()}}{}
numerical / functional integration

Perform integration on either numerical data or on a function.

The numerical intergration is based on given parameter x and y, based on
numpy function trap; while the functional integration is based on given
function and numpy function quad.
\begin{quote}\begin{description}
\item[{Parameters}] \leavevmode\begin{itemize}
\item {} 
\sphinxstyleliteralstrong{\sphinxupquote{x}} (\sphinxstyleliteralemphasis{\sphinxupquote{1D array}}) \textendash{} THe x axis values for numerical data, default None

\item {} 
\sphinxstyleliteralstrong{\sphinxupquote{y}} (\sphinxstyleliteralemphasis{\sphinxupquote{1D array}}) \textendash{} The y axis values for numerical data, default None

\item {} 
\sphinxstyleliteralstrong{\sphinxupquote{kind}} (\sphinxstyleliteralemphasis{\sphinxupquote{string}}) \textendash{} Specify the integration method, options are: ‘numerical’, ‘function’
default ‘numerical’

\item {} 
\sphinxstyleliteralstrong{\sphinxupquote{func}} (\sphinxstyleliteralemphasis{\sphinxupquote{function}}) \textendash{} The function to be integrated, default None

\item {} 
\sphinxstyleliteralstrong{\sphinxupquote{args}} \textendash{} arguments for function quad

\end{itemize}

\end{description}\end{quote}

\end{fulllineitems}



\section{scibeam.core.peak module}
\label{\detokenize{scibeam.core:module-scibeam.core.peak}}\label{\detokenize{scibeam.core:scibeam-core-peak-module}}\index{scibeam.core.peak (module)}\index{SeriesPeak (class in scibeam.core.peak)}

\begin{fulllineitems}
\phantomsection\label{\detokenize{scibeam.core:scibeam.core.peak.SeriesPeak}}\pysiglinewithargsret{\sphinxbfcode{\sphinxupquote{class }}\sphinxcode{\sphinxupquote{scibeam.core.peak.}}\sphinxbfcode{\sphinxupquote{SeriesPeak}}}{\emph{*args}, \emph{**kwargs}}{}
Bases: \sphinxcode{\sphinxupquote{pandas.core.series.Series}}

Peak analysis on 1D labeled / unlabeled data

Build on top of pandas.Series, this adds more methods on peak analysis for
pandas series data. The any class instance of SeriesPeak can still access
all pandas sereis methods.

By default, the indexes is treated as the time axis, while the values of
series is the data value.

Additionally, SeriesPeak is also designed as a mixin class which can be used
as a method chain in other pandas dataframe / sereis based data formats.
\index{self (scibeam.core.peak.SeriesPeak attribute)}

\begin{fulllineitems}
\phantomsection\label{\detokenize{scibeam.core:scibeam.core.peak.SeriesPeak.self}}\pysigline{\sphinxbfcode{\sphinxupquote{self}}}
\sphinxstyleemphasis{pandas series} \textendash{} pandas series data

\end{fulllineitems}

\index{\_\_init\_\_() (scibeam.core.peak.SeriesPeak method)}

\begin{fulllineitems}
\phantomsection\label{\detokenize{scibeam.core:scibeam.core.peak.SeriesPeak.__init__}}\pysiglinewithargsret{\sphinxbfcode{\sphinxupquote{\_\_init\_\_}}}{\emph{*args}, \emph{**kwargs}}{}
assign value to initialize SeriesPeak

The initlization of this class can be done exactly as one initlize
pandas series, for more information please pandas series documentation.

\end{fulllineitems}

\index{area() (scibeam.core.peak.SeriesPeak method)}

\begin{fulllineitems}
\phantomsection\label{\detokenize{scibeam.core:scibeam.core.peak.SeriesPeak.area}}\pysiglinewithargsret{\sphinxbfcode{\sphinxupquote{area}}}{\emph{gauss\_fit=False}, \emph{offset=False}}{}
\end{fulllineitems}

\index{autocrop() (scibeam.core.peak.SeriesPeak method)}

\begin{fulllineitems}
\phantomsection\label{\detokenize{scibeam.core:scibeam.core.peak.SeriesPeak.autocrop}}\pysiglinewithargsret{\sphinxbfcode{\sphinxupquote{autocrop}}}{\emph{n\_sigmas=4}, \emph{offset=False}}{}
\end{fulllineitems}

\index{fwhm() (scibeam.core.peak.SeriesPeak method)}

\begin{fulllineitems}
\phantomsection\label{\detokenize{scibeam.core:scibeam.core.peak.SeriesPeak.fwhm}}\pysiglinewithargsret{\sphinxbfcode{\sphinxupquote{fwhm}}}{\emph{gauss\_fit=False}, \emph{offset=False}}{}
Full-Width-Half-Maximium

Find the Full-Width-Half-Maximium (FWHM) of the peak, from
gaussian fitting or direction calculation.
\begin{quote}\begin{description}
\item[{Parameters}] \leavevmode\begin{itemize}
\item {} 
\sphinxstyleliteralstrong{\sphinxupquote{gauss\_fit}} (\sphinxstyleliteralemphasis{\sphinxupquote{bool}}) \textendash{} If true, fwhm is get from gaussian fitting
If false (default), fwhm is from direction calculation

\item {} 
\sphinxstyleliteralstrong{\sphinxupquote{offset}} (\sphinxstyleliteralemphasis{\sphinxupquote{bool}}) \textendash{} If True, the gaussian fitting will consider also fit the data
offset. If False (default), the fitting procedure will assume that
the data has 0 offset.

\end{itemize}

\item[{Returns}] \leavevmode
peak full-width-half-maximium value

\item[{Return type}] \leavevmode
float

\end{description}\end{quote}

\end{fulllineitems}

\index{gausFit() (scibeam.core.peak.SeriesPeak method)}

\begin{fulllineitems}
\phantomsection\label{\detokenize{scibeam.core:scibeam.core.peak.SeriesPeak.gausFit}}\pysiglinewithargsret{\sphinxbfcode{\sphinxupquote{gausFit}}}{\emph{plot=False}, \emph{offset=False}}{}
Fit series with gausssian function

This gasussian fit function assumes the time or x-axis values is given
by series index, while the measurement data or y-axis values is given by
the values of sereis.

Optionally the one can choose to plot the fitted gaussian curve together
with the raw data to visuallize the fitting property.
\begin{quote}\begin{description}
\item[{Parameters}] \leavevmode\begin{itemize}
\item {} 
\sphinxstyleliteralstrong{\sphinxupquote{plot}} (\sphinxstyleliteralemphasis{\sphinxupquote{bool}}) \textendash{} If True a plot will be generated with raw data and fitted gaussian
curve. Others no plot will be generated. Default False.

\item {} 
\sphinxstyleliteralstrong{\sphinxupquote{offset}} (\sphinxstyleliteralemphasis{\sphinxupquote{bool}}) \textendash{} If True, the gaussian fitting will consider also fit the data
offset. If False (default), the fitting procedure will assume that
the data has 0 offset.

\end{itemize}

\item[{Returns}] \leavevmode
\begin{itemize}
\item {} 
\sphinxstylestrong{popt} (\sphinxstyleemphasis{1D array}) \textendash{} optimized parameters of gaussian fitting.
{[}A, x0, sigma, y0(optional){]} Where
A: peak height of gaussian function
x0: peak center x coordinates
sigma: standard deviation
y0: offset. Only exist if parameter offset is set to be ‘True’

\item {} 
\sphinxstylestrong{pcov} (\sphinxstyleemphasis{2D array}) \textendash{} Covariance matrix of fitted parameters corresponding to popt

\end{itemize}


\end{description}\end{quote}

\end{fulllineitems}

\index{height() (scibeam.core.peak.SeriesPeak method)}

\begin{fulllineitems}
\phantomsection\label{\detokenize{scibeam.core:scibeam.core.peak.SeriesPeak.height}}\pysiglinewithargsret{\sphinxbfcode{\sphinxupquote{height}}}{\emph{gauss\_fit=False}, \emph{offset=False}}{}
calculate peak height

Calculated the peak height, either by gaussian fitting (if gauss\_fit
true), or simply return the maximium as the peak height (default)
\begin{quote}\begin{description}
\item[{Parameters}] \leavevmode\begin{itemize}
\item {} 
\sphinxstyleliteralstrong{\sphinxupquote{gauss\_fit}} (\sphinxstyleliteralemphasis{\sphinxupquote{bool}}) \textendash{} If true, the peak height is get by performing a gaussian fit
If false, simply the maximium value in the given data

\item {} 
\sphinxstyleliteralstrong{\sphinxupquote{offset}} (\sphinxstyleliteralemphasis{\sphinxupquote{bool}}) \textendash{} If True, the gaussian fitting will consider also fit the data
offset. If False (default), the fitting procedure will assume that
the data has 0 offset.

\end{itemize}

\item[{Returns}] \leavevmode
Peak height

\item[{Return type}] \leavevmode
float

\end{description}\end{quote}

\end{fulllineitems}

\index{idx() (scibeam.core.peak.SeriesPeak method)}

\begin{fulllineitems}
\phantomsection\label{\detokenize{scibeam.core:scibeam.core.peak.SeriesPeak.idx}}\pysiglinewithargsret{\sphinxbfcode{\sphinxupquote{idx}}}{\emph{gauss\_fit=False}, \emph{offset=False}}{}
find x-axis value corresponding to peak

This funciton is to locate the corresponding x corrdinate or ‘index’ of
the peak. Depend on the value of parameter ‘gauss\_fit’, the x coordinate
of peak can either come from the max value or gaussian fitting.

The index of series is treated as the x coordinate of the data.
\begin{quote}\begin{description}
\item[{Parameters}] \leavevmode\begin{itemize}
\item {} 
\sphinxstyleliteralstrong{\sphinxupquote{gauss\_fit}} (\sphinxstyleliteralemphasis{\sphinxupquote{bool}}) \textendash{} If true, the x coordinate corresponding to peak is get by performing
gaussian fitting on the data, as in member method gausFit.
If false, the maxmium value of data will be treated as the ‘peak’,
and its corresponding x coordinate will be returend.

\item {} 
\sphinxstyleliteralstrong{\sphinxupquote{offset}} (\sphinxstyleliteralemphasis{\sphinxupquote{bool}}) \textendash{} If True, the gaussian fitting will consider also fit the data
offset. If False (default), the fitting procedure will assume that
the data has 0 offset.

\end{itemize}

\item[{Returns}] \leavevmode
The x coordinate that corresponding to the peak

\item[{Return type}] \leavevmode
float

\end{description}\end{quote}

\end{fulllineitems}

\index{nidx() (scibeam.core.peak.SeriesPeak method)}

\begin{fulllineitems}
\phantomsection\label{\detokenize{scibeam.core:scibeam.core.peak.SeriesPeak.nidx}}\pysiglinewithargsret{\sphinxbfcode{\sphinxupquote{nidx}}}{\emph{gauss\_fit=False}, \emph{offset=False}}{}
number index of the peak

Similar to member method idx, this one returns the number index rather
than the real index, which means self.index{[}nidx{]} = idx

\end{fulllineitems}

\index{region() (scibeam.core.peak.SeriesPeak method)}

\begin{fulllineitems}
\phantomsection\label{\detokenize{scibeam.core:scibeam.core.peak.SeriesPeak.region}}\pysiglinewithargsret{\sphinxbfcode{\sphinxupquote{region}}}{\emph{n\_sigmas=4}, \emph{plot=False}, \emph{offset=False}}{}
Auto find the peak region

Locate the region where there exists a peak and return
the lower and upper bound index of the region.

\end{fulllineitems}

\index{sigma() (scibeam.core.peak.SeriesPeak method)}

\begin{fulllineitems}
\phantomsection\label{\detokenize{scibeam.core:scibeam.core.peak.SeriesPeak.sigma}}\pysiglinewithargsret{\sphinxbfcode{\sphinxupquote{sigma}}}{\emph{n\_sigmas=1}, \emph{gauss\_fit=False}, \emph{offset=False}}{}
Find peak width

Find the peak width, with specified mutiples of standard deviation.
The width can be obtained by literally calculate the full-width-half-max
or by gaussian fitting, depend on the parameter value ‘gauss\_fit’ to be
true of false.
\begin{quote}\begin{description}
\item[{Parameters}] \leavevmode\begin{itemize}
\item {} 
\sphinxstyleliteralstrong{\sphinxupquote{n\_sigmas}} (\sphinxstyleliteralemphasis{\sphinxupquote{integer}}) \textendash{} Multiplies of standard deviations of the peak width is wanted

\item {} 
\sphinxstyleliteralstrong{\sphinxupquote{gauss\_fit}} (\sphinxstyleliteralemphasis{\sphinxupquote{bool}}) \textendash{} If true, the peak width is obtained from gaussian fitting
If False (default), the peak width is calculated from literally
full-width-half-max.

\item {} 
\sphinxstyleliteralstrong{\sphinxupquote{offset}} (\sphinxstyleliteralemphasis{\sphinxupquote{bool}}) \textendash{} If True, the gaussian fitting will consider also fit the data
offset. If False (default), the fitting procedure will assume that
the data has 0 offset.

\end{itemize}

\item[{Returns}] \leavevmode
The peak width in terms of multiples of standard deviatioins

\item[{Return type}] \leavevmode
float

\end{description}\end{quote}

\end{fulllineitems}


\end{fulllineitems}

\index{FramePeak (class in scibeam.core.peak)}

\begin{fulllineitems}
\phantomsection\label{\detokenize{scibeam.core:scibeam.core.peak.FramePeak}}\pysiglinewithargsret{\sphinxbfcode{\sphinxupquote{class }}\sphinxcode{\sphinxupquote{scibeam.core.peak.}}\sphinxbfcode{\sphinxupquote{FramePeak}}}{\emph{*args}, \emph{**kwargs}}{}
Bases: \sphinxcode{\sphinxupquote{pandas.core.frame.DataFrame}}

Peak analysis on 1D labeled / unlabeled data
\index{area() (scibeam.core.peak.FramePeak method)}

\begin{fulllineitems}
\phantomsection\label{\detokenize{scibeam.core:scibeam.core.peak.FramePeak.area}}\pysiglinewithargsret{\sphinxbfcode{\sphinxupquote{area}}}{\emph{**kwargs}}{}
\end{fulllineitems}

\index{fwhm() (scibeam.core.peak.FramePeak method)}

\begin{fulllineitems}
\phantomsection\label{\detokenize{scibeam.core:scibeam.core.peak.FramePeak.fwhm}}\pysiglinewithargsret{\sphinxbfcode{\sphinxupquote{fwhm}}}{\emph{**kwargs}}{}
\end{fulllineitems}

\index{height() (scibeam.core.peak.FramePeak method)}

\begin{fulllineitems}
\phantomsection\label{\detokenize{scibeam.core:scibeam.core.peak.FramePeak.height}}\pysiglinewithargsret{\sphinxbfcode{\sphinxupquote{height}}}{\emph{**kwargs}}{}
\end{fulllineitems}

\index{idx() (scibeam.core.peak.FramePeak method)}

\begin{fulllineitems}
\phantomsection\label{\detokenize{scibeam.core:scibeam.core.peak.FramePeak.idx}}\pysiglinewithargsret{\sphinxbfcode{\sphinxupquote{idx}}}{\emph{**kwargs}}{}
\end{fulllineitems}

\index{nidx() (scibeam.core.peak.FramePeak method)}

\begin{fulllineitems}
\phantomsection\label{\detokenize{scibeam.core:scibeam.core.peak.FramePeak.nidx}}\pysiglinewithargsret{\sphinxbfcode{\sphinxupquote{nidx}}}{\emph{**kwargs}}{}
\end{fulllineitems}

\index{region() (scibeam.core.peak.FramePeak method)}

\begin{fulllineitems}
\phantomsection\label{\detokenize{scibeam.core:scibeam.core.peak.FramePeak.region}}\pysiglinewithargsret{\sphinxbfcode{\sphinxupquote{region}}}{\emph{**kwargs}}{}
\end{fulllineitems}

\index{sigma() (scibeam.core.peak.FramePeak method)}

\begin{fulllineitems}
\phantomsection\label{\detokenize{scibeam.core:scibeam.core.peak.FramePeak.sigma}}\pysiglinewithargsret{\sphinxbfcode{\sphinxupquote{sigma}}}{\emph{**kwargs}}{}
\end{fulllineitems}


\end{fulllineitems}



\section{scibeam.core.plot module}
\label{\detokenize{scibeam.core:module-scibeam.core.plot}}\label{\detokenize{scibeam.core:scibeam-core-plot-module}}\index{scibeam.core.plot (module)}\index{PlotTOFFrame (class in scibeam.core.plot)}

\begin{fulllineitems}
\phantomsection\label{\detokenize{scibeam.core:scibeam.core.plot.PlotTOFFrame}}\pysiglinewithargsret{\sphinxbfcode{\sphinxupquote{class }}\sphinxcode{\sphinxupquote{scibeam.core.plot.}}\sphinxbfcode{\sphinxupquote{PlotTOFFrame}}}{\emph{dataframe}, \emph{lowerBound=None}, \emph{upperBound=None}, \emph{index\_label=None}, \emph{column\_label=None}}{}
Bases: \sphinxcode{\sphinxupquote{object}}

Plot dataframe with time as index and another numerical variable as
column labels
\index{contour() (scibeam.core.plot.PlotTOFFrame method)}

\begin{fulllineitems}
\phantomsection\label{\detokenize{scibeam.core:scibeam.core.plot.PlotTOFFrame.contour}}\pysiglinewithargsret{\sphinxbfcode{\sphinxupquote{contour}}}{\emph{n\_contours=5}, \emph{n\_sigma=2}, \emph{xlabel='time'}, \emph{ylabel='value'}, \emph{title='contour plot'}, \emph{label=None}, \emph{ax=None}, \emph{image=False}, \emph{**kwargs}}{}
contour plots for 2D self.data

\end{fulllineitems}

\index{contourf() (scibeam.core.plot.PlotTOFFrame method)}

\begin{fulllineitems}
\phantomsection\label{\detokenize{scibeam.core:scibeam.core.plot.PlotTOFFrame.contourf}}\pysiglinewithargsret{\sphinxbfcode{\sphinxupquote{contourf}}}{\emph{n\_contours=5}, \emph{n\_sigma=2}, \emph{xlabel='time'}, \emph{ylabel='value'}, \emph{title='contour plot'}, \emph{label=None}, \emph{ax=None}, \emph{**kwargs}}{}
contourf plots for 2D self.data

\end{fulllineitems}

\index{data (scibeam.core.plot.PlotTOFFrame attribute)}

\begin{fulllineitems}
\phantomsection\label{\detokenize{scibeam.core:scibeam.core.plot.PlotTOFFrame.data}}\pysigline{\sphinxbfcode{\sphinxupquote{data}}}
\end{fulllineitems}

\index{image() (scibeam.core.plot.PlotTOFFrame method)}

\begin{fulllineitems}
\phantomsection\label{\detokenize{scibeam.core:scibeam.core.plot.PlotTOFFrame.image}}\pysiglinewithargsret{\sphinxbfcode{\sphinxupquote{image}}}{\emph{sideplots=True}, \emph{contour=False}, \emph{**kwargs}}{}
image plot of tof data measured multiplot positions

\end{fulllineitems}


\end{fulllineitems}

\index{PlotTOFSeries (class in scibeam.core.plot)}

\begin{fulllineitems}
\phantomsection\label{\detokenize{scibeam.core:scibeam.core.plot.PlotTOFSeries}}\pysiglinewithargsret{\sphinxbfcode{\sphinxupquote{class }}\sphinxcode{\sphinxupquote{scibeam.core.plot.}}\sphinxbfcode{\sphinxupquote{PlotTOFSeries}}}{\emph{dataseries}, \emph{lowerBound=None}, \emph{upperBound=None}, \emph{index\_label=None}, \emph{column\_name=None}}{}
Bases: \sphinxcode{\sphinxupquote{object}}

Plot dataframe with time as index and another numerical variable as
column labels
\index{data (scibeam.core.plot.PlotTOFSeries attribute)}

\begin{fulllineitems}
\phantomsection\label{\detokenize{scibeam.core:scibeam.core.plot.PlotTOFSeries.data}}\pysigline{\sphinxbfcode{\sphinxupquote{data}}}
\end{fulllineitems}

\index{plot() (scibeam.core.plot.PlotTOFSeries method)}

\begin{fulllineitems}
\phantomsection\label{\detokenize{scibeam.core:scibeam.core.plot.PlotTOFSeries.plot}}\pysiglinewithargsret{\sphinxbfcode{\sphinxupquote{plot}}}{\emph{ax=None}, \emph{gauss\_fit=True}, \emph{gauss\_fit\_offset=0}, \emph{print\_fit\_params=True}, \emph{title=None}, \emph{xlabel=None}, \emph{ylabel=None}, \emph{label=None}, \emph{params\_digits=3}, \emph{**kwargs}}{}
\end{fulllineitems}


\end{fulllineitems}



\section{scibeam.core.regexp module}
\label{\detokenize{scibeam.core:module-scibeam.core.regexp}}\label{\detokenize{scibeam.core:scibeam-core-regexp-module}}\index{scibeam.core.regexp (module)}\index{RegMatch (class in scibeam.core.regexp)}

\begin{fulllineitems}
\phantomsection\label{\detokenize{scibeam.core:scibeam.core.regexp.RegMatch}}\pysiglinewithargsret{\sphinxbfcode{\sphinxupquote{class }}\sphinxcode{\sphinxupquote{scibeam.core.regexp.}}\sphinxbfcode{\sphinxupquote{RegMatch}}}{\emph{regStr}}{}
Bases: \sphinxcode{\sphinxupquote{object}}
\index{match() (scibeam.core.regexp.RegMatch method)}

\begin{fulllineitems}
\phantomsection\label{\detokenize{scibeam.core:scibeam.core.regexp.RegMatch.match}}\pysiglinewithargsret{\sphinxbfcode{\sphinxupquote{match}}}{\emph{strings}, \emph{group=1}, \emph{asNumber=True}}{}
Match a single or list of regularizations to a single or list of strings
Return as a dictionary

\end{fulllineitems}

\index{matchFolder() (scibeam.core.regexp.RegMatch method)}

\begin{fulllineitems}
\phantomsection\label{\detokenize{scibeam.core:scibeam.core.regexp.RegMatch.matchFolder}}\pysiglinewithargsret{\sphinxbfcode{\sphinxupquote{matchFolder}}}{\emph{folder\_path}, \emph{asNumber=True}, \emph{group=1}}{}
Match files in the folder content with self.regex
if two regex are in the self.regex, then the match is done
in a recursive way, that first regex get matched, and the 2nd
regex is applied to the match result from the first one.

\end{fulllineitems}

\index{regex (scibeam.core.regexp.RegMatch attribute)}

\begin{fulllineitems}
\phantomsection\label{\detokenize{scibeam.core:scibeam.core.regexp.RegMatch.regex}}\pysigline{\sphinxbfcode{\sphinxupquote{regex}}}
\end{fulllineitems}

\index{single\_regex\_match() (scibeam.core.regexp.RegMatch static method)}

\begin{fulllineitems}
\phantomsection\label{\detokenize{scibeam.core:scibeam.core.regexp.RegMatch.single_regex_match}}\pysiglinewithargsret{\sphinxbfcode{\sphinxupquote{static }}\sphinxbfcode{\sphinxupquote{single\_regex\_match}}}{\emph{regStr}, \emph{strings}, \emph{group=1}, \emph{asNumber=False}}{}
Match python regex pattern in a given string or list of strings
Based on python re package and uses group to locate the value

returns pairs of (value, string) matched pairs

\end{fulllineitems}


\end{fulllineitems}



\section{scibeam.core.tofframe module}
\label{\detokenize{scibeam.core:module-scibeam.core.tofframe}}\label{\detokenize{scibeam.core:scibeam-core-tofframe-module}}\index{scibeam.core.tofframe (module)}\index{TOFFrame (class in scibeam.core.tofframe)}

\begin{fulllineitems}
\phantomsection\label{\detokenize{scibeam.core:scibeam.core.tofframe.TOFFrame}}\pysiglinewithargsret{\sphinxbfcode{\sphinxupquote{class }}\sphinxcode{\sphinxupquote{scibeam.core.tofframe.}}\sphinxbfcode{\sphinxupquote{TOFFrame}}}{\emph{*args}, \emph{**kwargs}}{}
Bases: \sphinxcode{\sphinxupquote{pandas.core.frame.DataFrame}}

Time-Of-Flight (TOF) DataFrame

Subclassing pandas.DataFrame with extral methods / properties for time-series analysis
\begin{quote}\begin{description}
\item[{Parameters}] \leavevmode\begin{itemize}
\item {} 
\sphinxstyleliteralstrong{\sphinxupquote{data}} (\sphinxstyleliteralemphasis{\sphinxupquote{numpy ndarray}}\sphinxstyleliteralemphasis{\sphinxupquote{ (}}\sphinxstyleliteralemphasis{\sphinxupquote{structured}}\sphinxstyleliteralemphasis{\sphinxupquote{ or }}\sphinxstyleliteralemphasis{\sphinxupquote{homogeneous}}\sphinxstyleliteralemphasis{\sphinxupquote{)}}\sphinxstyleliteralemphasis{\sphinxupquote{, }}\sphinxstyleliteralemphasis{\sphinxupquote{dict}}\sphinxstyleliteralemphasis{\sphinxupquote{, or }}\sphinxstyleliteralemphasis{\sphinxupquote{DataFrame}}) \textendash{} Dict can contain Series, arrays, constants, or list-like objectsSingle time-of-flight data analysis
Value of measurement, e.g. voltage, current, arbiturary unit signel, shape(len(labels), len(times))

\item {} 
\sphinxstyleliteralstrong{\sphinxupquote{index}} (\sphinxstyleliteralemphasis{\sphinxupquote{numpy ndarray}}\sphinxstyleliteralemphasis{\sphinxupquote{, }}\sphinxstyleliteralemphasis{\sphinxupquote{iterables}}) \textendash{} Time axis for time-of-flight

\item {} 
\sphinxstyleliteralstrong{\sphinxupquote{columns}} (\sphinxstyleliteralemphasis{\sphinxupquote{str}}\sphinxstyleliteralemphasis{\sphinxupquote{, }}\sphinxstyleliteralemphasis{\sphinxupquote{int}}\sphinxstyleliteralemphasis{\sphinxupquote{, or }}\sphinxstyleliteralemphasis{\sphinxupquote{float}}) \textendash{} label of different tof measurement, e.g. pressure, temperature, etc

\end{itemize}

\end{description}\end{quote}
\index{find\_time\_idx() (scibeam.core.tofframe.TOFFrame static method)}

\begin{fulllineitems}
\phantomsection\label{\detokenize{scibeam.core:scibeam.core.tofframe.TOFFrame.find_time_idx}}\pysiglinewithargsret{\sphinxbfcode{\sphinxupquote{static }}\sphinxbfcode{\sphinxupquote{find\_time\_idx}}}{\emph{time}, \emph{*args}}{}
Generator of time index for a given time value
args: can be 1,2,3, or {[}1,2{]} or {[}1,2,3{]}

\end{fulllineitems}

\index{from\_file() (scibeam.core.tofframe.TOFFrame class method)}

\begin{fulllineitems}
\phantomsection\label{\detokenize{scibeam.core:scibeam.core.tofframe.TOFFrame.from_file}}\pysiglinewithargsret{\sphinxbfcode{\sphinxupquote{classmethod }}\sphinxbfcode{\sphinxupquote{from\_file}}}{\emph{filePath}, \emph{lowerBound=None}, \emph{upperBound=None}, \emph{removeOffset=True}, \emph{offset\_margin\_how='outer'}, \emph{offset\_margin\_size=20}, \emph{skiprows=0}, \emph{sep='\textbackslash{}t'}}{}
Generate TOFFrame object from a single given file

\end{fulllineitems}

\index{from\_matchResult() (scibeam.core.tofframe.TOFFrame class method)}

\begin{fulllineitems}
\phantomsection\label{\detokenize{scibeam.core:scibeam.core.tofframe.TOFFrame.from_matchResult}}\pysiglinewithargsret{\sphinxbfcode{\sphinxupquote{classmethod }}\sphinxbfcode{\sphinxupquote{from\_matchResult}}}{\emph{path}, \emph{matchDict}, \emph{lowerBound=None}, \emph{upperBound=None}, \emph{removeOffset=True}, \emph{offset\_margin\_how='outer'}, \emph{offset\_margin\_size=20}, \emph{skiprows=0}, \emph{sep='\textbackslash{}t'}}{}
Creat TOFFrame from a RegMatch resutl dictionary

\end{fulllineitems}

\index{from\_path() (scibeam.core.tofframe.TOFFrame class method)}

\begin{fulllineitems}
\phantomsection\label{\detokenize{scibeam.core:scibeam.core.tofframe.TOFFrame.from_path}}\pysiglinewithargsret{\sphinxbfcode{\sphinxupquote{classmethod }}\sphinxbfcode{\sphinxupquote{from\_path}}}{\emph{path}, \emph{regStr}, \emph{lowerBound=None}, \emph{upperBound=None}, \emph{removeOffset=True}, \emph{offset\_margin\_how='outer'}, \emph{offset\_margin\_size=20}, \emph{skiprows=0}, \emph{sep='\textbackslash{}t'}}{}
Buid TOFFrome instance from given file folder
Current only works for ‘        ‘ seperated txt and lvm file

\end{fulllineitems}

\index{inch\_to\_mm() (scibeam.core.tofframe.TOFFrame method)}

\begin{fulllineitems}
\phantomsection\label{\detokenize{scibeam.core:scibeam.core.tofframe.TOFFrame.inch_to_mm}}\pysiglinewithargsret{\sphinxbfcode{\sphinxupquote{inch\_to\_mm}}}{\emph{**kwargs}}{}
\end{fulllineitems}

\index{microsec\_to\_sec() (scibeam.core.tofframe.TOFFrame method)}

\begin{fulllineitems}
\phantomsection\label{\detokenize{scibeam.core:scibeam.core.tofframe.TOFFrame.microsec_to_sec}}\pysiglinewithargsret{\sphinxbfcode{\sphinxupquote{microsec\_to\_sec}}}{\emph{**kwargs}}{}
\end{fulllineitems}

\index{mm\_to\_inch() (scibeam.core.tofframe.TOFFrame method)}

\begin{fulllineitems}
\phantomsection\label{\detokenize{scibeam.core:scibeam.core.tofframe.TOFFrame.mm_to_inch}}\pysiglinewithargsret{\sphinxbfcode{\sphinxupquote{mm\_to\_inch}}}{\emph{**kwargs}}{}
\end{fulllineitems}

\index{peak (scibeam.core.tofframe.TOFFrame attribute)}

\begin{fulllineitems}
\phantomsection\label{\detokenize{scibeam.core:scibeam.core.tofframe.TOFFrame.peak}}\pysigline{\sphinxbfcode{\sphinxupquote{peak}}}
alias of {\hyperref[\detokenize{scibeam.core:scibeam.core.peak.FramePeak}]{\sphinxcrossref{\sphinxcode{\sphinxupquote{scibeam.core.peak.FramePeak}}}}}

\end{fulllineitems}

\index{plot2d (scibeam.core.tofframe.TOFFrame attribute)}

\begin{fulllineitems}
\phantomsection\label{\detokenize{scibeam.core:scibeam.core.tofframe.TOFFrame.plot2d}}\pysigline{\sphinxbfcode{\sphinxupquote{plot2d}}}
alias of {\hyperref[\detokenize{scibeam.core:scibeam.core.plot.PlotTOFFrame}]{\sphinxcrossref{\sphinxcode{\sphinxupquote{scibeam.core.plot.PlotTOFFrame}}}}}

\end{fulllineitems}

\index{reduce() (scibeam.core.tofframe.TOFFrame method)}

\begin{fulllineitems}
\phantomsection\label{\detokenize{scibeam.core:scibeam.core.tofframe.TOFFrame.reduce}}\pysiglinewithargsret{\sphinxbfcode{\sphinxupquote{reduce}}}{\emph{**kwargs}}{}
\end{fulllineitems}

\index{remove\_data\_offset() (scibeam.core.tofframe.TOFFrame static method)}

\begin{fulllineitems}
\phantomsection\label{\detokenize{scibeam.core:scibeam.core.tofframe.TOFFrame.remove_data_offset}}\pysiglinewithargsret{\sphinxbfcode{\sphinxupquote{static }}\sphinxbfcode{\sphinxupquote{remove\_data\_offset}}}{\emph{data}, \emph{lowerBoundIdx=None}, \emph{upperBoundIdx=None}, \emph{how='outer'}, \emph{margin\_size=10}}{}
remove offset in 1D array data

\end{fulllineitems}

\index{sec\_to\_microsec() (scibeam.core.tofframe.TOFFrame method)}

\begin{fulllineitems}
\phantomsection\label{\detokenize{scibeam.core:scibeam.core.tofframe.TOFFrame.sec_to_microsec}}\pysiglinewithargsret{\sphinxbfcode{\sphinxupquote{sec\_to\_microsec}}}{\emph{**kwargs}}{}
\end{fulllineitems}

\index{selectTimeRange() (scibeam.core.tofframe.TOFFrame method)}

\begin{fulllineitems}
\phantomsection\label{\detokenize{scibeam.core:scibeam.core.tofframe.TOFFrame.selectTimeRange}}\pysiglinewithargsret{\sphinxbfcode{\sphinxupquote{selectTimeRange}}}{\emph{**kwargs}}{}
\end{fulllineitems}

\index{selectTimeSlice() (scibeam.core.tofframe.TOFFrame method)}

\begin{fulllineitems}
\phantomsection\label{\detokenize{scibeam.core:scibeam.core.tofframe.TOFFrame.selectTimeSlice}}\pysiglinewithargsret{\sphinxbfcode{\sphinxupquote{selectTimeSlice}}}{\emph{**kwargs}}{}
\end{fulllineitems}

\index{sum() (scibeam.core.tofframe.TOFFrame method)}

\begin{fulllineitems}
\phantomsection\label{\detokenize{scibeam.core:scibeam.core.tofframe.TOFFrame.sum}}\pysiglinewithargsret{\sphinxbfcode{\sphinxupquote{sum}}}{\emph{**kwargs}}{}
\end{fulllineitems}


\end{fulllineitems}

\index{read\_folder() (in module scibeam.core.tofframe)}

\begin{fulllineitems}
\phantomsection\label{\detokenize{scibeam.core:scibeam.core.tofframe.read_folder}}\pysiglinewithargsret{\sphinxcode{\sphinxupquote{scibeam.core.tofframe.}}\sphinxbfcode{\sphinxupquote{read\_folder}}}{\emph{path}, \emph{regStr}, \emph{lowerBound=None}, \emph{upperBound=None}, \emph{removeOffset=True}, \emph{offset\_margin\_how='outer'}, \emph{offset\_margin\_size=20}, \emph{skiprows=0}, \emph{sep='\textbackslash{}t'}}{}
Create TOFFrame class instance by reading in group of files in a folder matched by regex
\begin{quote}\begin{description}
\item[{Parameters}] \leavevmode\begin{itemize}
\item {} 
\sphinxstyleliteralstrong{\sphinxupquote{path}} (\sphinxstyleliteralemphasis{\sphinxupquote{str}}) \textendash{} folder path, linux style or windows style as “raw string”, e.g. r”C:UserDocumentFolderName”

\item {} 
\sphinxstyleliteralstrong{\sphinxupquote{lowerBound}} (\sphinxstyleliteralemphasis{\sphinxupquote{int}}\sphinxstyleliteralemphasis{\sphinxupquote{ or }}\sphinxstyleliteralemphasis{\sphinxupquote{float}}) \textendash{} time axis lower boundrary limit for data

\item {} 
\sphinxstyleliteralstrong{\sphinxupquote{upperBound}} (\sphinxstyleliteralemphasis{\sphinxupquote{int}}\sphinxstyleliteralemphasis{\sphinxupquote{ or }}\sphinxstyleliteralemphasis{\sphinxupquote{float}}) \textendash{} time axis upper boundrary limit for data

\item {} 
\sphinxstyleliteralstrong{\sphinxupquote{removeOffset}} (\sphinxstyleliteralemphasis{\sphinxupquote{bool}}) \textendash{} if True (default) remove data offset (set floor to 0 in no-signal region)

\item {} 
\sphinxstyleliteralstrong{\sphinxupquote{offset\_margin\_how}} (\sphinxstyleliteralemphasis{\sphinxupquote{\{"outer"}}\sphinxstyleliteralemphasis{\sphinxupquote{, }}\sphinxstyleliteralemphasis{\sphinxupquote{"outer left"}}\sphinxstyleliteralemphasis{\sphinxupquote{, }}\sphinxstyleliteralemphasis{\sphinxupquote{"out right"}}\sphinxstyleliteralemphasis{\sphinxupquote{, }}\sphinxstyleliteralemphasis{\sphinxupquote{"inner"}}\sphinxstyleliteralemphasis{\sphinxupquote{, }}\sphinxstyleliteralemphasis{\sphinxupquote{"inner left"}}\sphinxstyleliteralemphasis{\sphinxupquote{, }}\sphinxstyleliteralemphasis{\sphinxupquote{"inner right"\}}}\sphinxstyleliteralemphasis{\sphinxupquote{, }}\sphinxstyleliteralemphasis{\sphinxupquote{default "outer"}}) \textendash{} 
Specify the way to handle offset margin, offset floor value is calculated by averaging the
value in a given range relative to data lower and upper boundrary, with avaliable options:
\begin{itemize}
\item {} 
”outer” (default):  from both left and right side out of the {[}lowerBound, upperBound{]} region

\item {} 
”outer left”: like “outer” but from only left side

\item {} 
”outer right”: like “outer” but from only right side

\item {} 
”inner”: from both left and right side inside of the {[}lowerBound, upperBound{]} region

\item {} 
”inner left”: like “inner” but from only left side

\item {} 
”inner right”: like “inner” but from only left side

\end{itemize}


\item {} 
\sphinxstyleliteralstrong{\sphinxupquote{offset\_margin\_size}} (\sphinxstyleliteralemphasis{\sphinxupquote{int}}) \textendash{} Number of values to use for averaging when calculating offset

\item {} 
\sphinxstyleliteralstrong{\sphinxupquote{skiprows}} (\sphinxstyleliteralemphasis{\sphinxupquote{int}}) \textendash{} number of rows to skip when read in data

\item {} 
\sphinxstyleliteralstrong{\sphinxupquote{sep}} (\sphinxstyleliteralemphasis{\sphinxupquote{str}}\sphinxstyleliteralemphasis{\sphinxupquote{, }}\sphinxstyleliteralemphasis{\sphinxupquote{defult "  "}}) \textendash{} seperator for columns in the data file

\item {} 
\sphinxstyleliteralstrong{\sphinxupquote{Returns}} \textendash{} 

\item {} 
\sphinxstyleliteralstrong{\sphinxupquote{-{-}-{-}-{-}-{-}}} \textendash{} 

\item {} 
\sphinxstyleliteralstrong{\sphinxupquote{of class TOFFrame}} (\sphinxstyleliteralemphasis{\sphinxupquote{Instance}}) \textendash{} 

\end{itemize}

\end{description}\end{quote}

\end{fulllineitems}

\index{read\_regexp\_match() (in module scibeam.core.tofframe)}

\begin{fulllineitems}
\phantomsection\label{\detokenize{scibeam.core:scibeam.core.tofframe.read_regexp_match}}\pysiglinewithargsret{\sphinxcode{\sphinxupquote{scibeam.core.tofframe.}}\sphinxbfcode{\sphinxupquote{read\_regexp\_match}}}{\emph{path}, \emph{matchDict}, \emph{lowerBound=None}, \emph{upperBound=None}, \emph{removeOffset=True}, \emph{offset\_margin\_how='outer'}, \emph{offset\_margin\_size=20}, \emph{skiprows=0}, \emph{sep='\textbackslash{}t'}}{}
Create instance of TOFFrame from regular expression match result dictionary
using scibeam class RegMatch
\begin{quote}\begin{description}
\item[{Parameters}] \leavevmode\begin{itemize}
\item {} 
\sphinxstyleliteralstrong{\sphinxupquote{path}} (\sphinxstyleliteralemphasis{\sphinxupquote{str}}) \textendash{} path of the targeted data folder

\item {} 
\sphinxstyleliteralstrong{\sphinxupquote{matchDict}} (\sphinxstyleliteralemphasis{\sphinxupquote{dictionary}}) \textendash{} result dictionary form scibeam.regexp.RegMatch, or user specified
dictionary with key as measurement label, value as file name string

\item {} 
\sphinxstyleliteralstrong{\sphinxupquote{lowerBound}} (\sphinxstyleliteralemphasis{\sphinxupquote{int}}\sphinxstyleliteralemphasis{\sphinxupquote{ or }}\sphinxstyleliteralemphasis{\sphinxupquote{float}}) \textendash{} time axis lower boundrary limit for data

\item {} 
\sphinxstyleliteralstrong{\sphinxupquote{upperBound}} (\sphinxstyleliteralemphasis{\sphinxupquote{int}}\sphinxstyleliteralemphasis{\sphinxupquote{ or }}\sphinxstyleliteralemphasis{\sphinxupquote{float}}) \textendash{} time axis upper boundrary limit for data

\item {} 
\sphinxstyleliteralstrong{\sphinxupquote{removeOffset}} (\sphinxstyleliteralemphasis{\sphinxupquote{bool}}) \textendash{} if True (default) remove data offset (set floor to 0 in no-signal region)

\item {} 
\sphinxstyleliteralstrong{\sphinxupquote{offset\_margin\_how}} (\sphinxstyleliteralemphasis{\sphinxupquote{\{"outer"}}\sphinxstyleliteralemphasis{\sphinxupquote{, }}\sphinxstyleliteralemphasis{\sphinxupquote{"outer left"}}\sphinxstyleliteralemphasis{\sphinxupquote{, }}\sphinxstyleliteralemphasis{\sphinxupquote{"out right"}}\sphinxstyleliteralemphasis{\sphinxupquote{, }}\sphinxstyleliteralemphasis{\sphinxupquote{"inner"}}\sphinxstyleliteralemphasis{\sphinxupquote{, }}\sphinxstyleliteralemphasis{\sphinxupquote{"inner left"}}\sphinxstyleliteralemphasis{\sphinxupquote{, }}\sphinxstyleliteralemphasis{\sphinxupquote{"inner right"\}}}\sphinxstyleliteralemphasis{\sphinxupquote{, }}\sphinxstyleliteralemphasis{\sphinxupquote{default "outer"}}) \textendash{} 
Specify the way to handle offset margin, offset floor value is calculated by averaging the
value in a given range relative to data lower and upper boundrary, with avaliable options:
\begin{itemize}
\item {} 
”outer” (default):  from both left and right side out of the {[}lowerBound, upperBound{]} region

\item {} 
”outer left”: like “outer” but from only left side

\item {} 
”outer right”: like “outer” but from only right side

\item {} 
”inner”: from both left and right side inside of the {[}lowerBound, upperBound{]} region

\item {} 
”inner left”: like “inner” but from only left side

\item {} 
”inner right”: like “inner” but from only left side

\end{itemize}


\item {} 
\sphinxstyleliteralstrong{\sphinxupquote{offset\_margin\_size}} (\sphinxstyleliteralemphasis{\sphinxupquote{int}}) \textendash{} Number of values to use for averaging when calculating offset

\item {} 
\sphinxstyleliteralstrong{\sphinxupquote{skiprows}} (\sphinxstyleliteralemphasis{\sphinxupquote{int}}) \textendash{} number of rows to skip when read in data

\item {} 
\sphinxstyleliteralstrong{\sphinxupquote{sep}} (\sphinxstyleliteralemphasis{\sphinxupquote{str}}\sphinxstyleliteralemphasis{\sphinxupquote{, }}\sphinxstyleliteralemphasis{\sphinxupquote{defult "  "}}) \textendash{} seperator for columns in the data file

\end{itemize}

\item[{Returns}] \leavevmode


\item[{Return type}] \leavevmode
Instance of TOFFrame

\end{description}\end{quote}

\end{fulllineitems}



\section{scibeam.core.tofseries module}
\label{\detokenize{scibeam.core:module-scibeam.core.tofseries}}\label{\detokenize{scibeam.core:scibeam-core-tofseries-module}}\index{scibeam.core.tofseries (module)}\index{TOFSeries (class in scibeam.core.tofseries)}

\begin{fulllineitems}
\phantomsection\label{\detokenize{scibeam.core:scibeam.core.tofseries.TOFSeries}}\pysiglinewithargsret{\sphinxbfcode{\sphinxupquote{class }}\sphinxcode{\sphinxupquote{scibeam.core.tofseries.}}\sphinxbfcode{\sphinxupquote{TOFSeries}}}{\emph{*args}, \emph{**kwargs}}{}
Bases: \sphinxcode{\sphinxupquote{pandas.core.series.Series}}
\index{find\_time\_idx() (scibeam.core.tofseries.TOFSeries static method)}

\begin{fulllineitems}
\phantomsection\label{\detokenize{scibeam.core:scibeam.core.tofseries.TOFSeries.find_time_idx}}\pysiglinewithargsret{\sphinxbfcode{\sphinxupquote{static }}\sphinxbfcode{\sphinxupquote{find\_time\_idx}}}{\emph{time}, \emph{*args}}{}
\end{fulllineitems}

\index{from\_file() (scibeam.core.tofseries.TOFSeries class method)}

\begin{fulllineitems}
\phantomsection\label{\detokenize{scibeam.core:scibeam.core.tofseries.TOFSeries.from_file}}\pysiglinewithargsret{\sphinxbfcode{\sphinxupquote{classmethod }}\sphinxbfcode{\sphinxupquote{from\_file}}}{\emph{file\_path}, \emph{lowerBound=None}, \emph{upperBound=None}, \emph{removeOffset=True}, \emph{cols=2}, \emph{usecols=None}, \emph{offset\_margin\_how='outer'}, \emph{offset\_margin\_size=20}, \emph{skiprows=0}, \emph{sep='\textbackslash{}t'}}{}
Buid TOF instance from given file
Current only works for ‘        ‘ seperated txt and lvm file

\end{fulllineitems}

\index{gausCenter() (scibeam.core.tofseries.TOFSeries method)}

\begin{fulllineitems}
\phantomsection\label{\detokenize{scibeam.core:scibeam.core.tofseries.TOFSeries.gausCenter}}\pysiglinewithargsret{\sphinxbfcode{\sphinxupquote{gausCenter}}}{\emph{offset=False}}{}
gaus fit center

\end{fulllineitems}

\index{gausFit() (scibeam.core.tofseries.TOFSeries method)}

\begin{fulllineitems}
\phantomsection\label{\detokenize{scibeam.core:scibeam.core.tofseries.TOFSeries.gausFit}}\pysiglinewithargsret{\sphinxbfcode{\sphinxupquote{gausFit}}}{\emph{offset=False}}{}
1D gauss fit

\end{fulllineitems}

\index{gausStd() (scibeam.core.tofseries.TOFSeries method)}

\begin{fulllineitems}
\phantomsection\label{\detokenize{scibeam.core:scibeam.core.tofseries.TOFSeries.gausStd}}\pysiglinewithargsret{\sphinxbfcode{\sphinxupquote{gausStd}}}{\emph{offset=False}}{}
gaus fit std

\end{fulllineitems}

\index{peak (scibeam.core.tofseries.TOFSeries attribute)}

\begin{fulllineitems}
\phantomsection\label{\detokenize{scibeam.core:scibeam.core.tofseries.TOFSeries.peak}}\pysigline{\sphinxbfcode{\sphinxupquote{peak}}}
alias of {\hyperref[\detokenize{scibeam.core:scibeam.core.peak.SeriesPeak}]{\sphinxcrossref{\sphinxcode{\sphinxupquote{scibeam.core.peak.SeriesPeak}}}}}

\end{fulllineitems}

\index{plot1d (scibeam.core.tofseries.TOFSeries attribute)}

\begin{fulllineitems}
\phantomsection\label{\detokenize{scibeam.core:scibeam.core.tofseries.TOFSeries.plot1d}}\pysigline{\sphinxbfcode{\sphinxupquote{plot1d}}}
alias of {\hyperref[\detokenize{scibeam.core:scibeam.core.plot.PlotTOFSeries}]{\sphinxcrossref{\sphinxcode{\sphinxupquote{scibeam.core.plot.PlotTOFSeries}}}}}

\end{fulllineitems}

\index{remove\_data\_offset() (scibeam.core.tofseries.TOFSeries static method)}

\begin{fulllineitems}
\phantomsection\label{\detokenize{scibeam.core:scibeam.core.tofseries.TOFSeries.remove_data_offset}}\pysiglinewithargsret{\sphinxbfcode{\sphinxupquote{static }}\sphinxbfcode{\sphinxupquote{remove\_data\_offset}}}{\emph{data}, \emph{lowerBoundIdx=None}, \emph{upperBoundIdx=None}, \emph{how='outer'}, \emph{margin\_size=10}}{}
remove offset in 1D array data

\end{fulllineitems}

\index{sec\_to\_microsec() (scibeam.core.tofseries.TOFSeries method)}

\begin{fulllineitems}
\phantomsection\label{\detokenize{scibeam.core:scibeam.core.tofseries.TOFSeries.sec_to_microsec}}\pysiglinewithargsret{\sphinxbfcode{\sphinxupquote{sec\_to\_microsec}}}{\emph{offset\_sec=0}, \emph{inplace=False}}{}
convert seconds in index to microseconds

\end{fulllineitems}

\index{selectTimeRange() (scibeam.core.tofseries.TOFSeries method)}

\begin{fulllineitems}
\phantomsection\label{\detokenize{scibeam.core:scibeam.core.tofseries.TOFSeries.selectTimeRange}}\pysiglinewithargsret{\sphinxbfcode{\sphinxupquote{selectTimeRange}}}{\emph{**kwargs}}{}
\end{fulllineitems}

\index{selectTimeSlice() (scibeam.core.tofseries.TOFSeries method)}

\begin{fulllineitems}
\phantomsection\label{\detokenize{scibeam.core:scibeam.core.tofseries.TOFSeries.selectTimeSlice}}\pysiglinewithargsret{\sphinxbfcode{\sphinxupquote{selectTimeSlice}}}{\emph{**kwargs}}{}
\end{fulllineitems}


\end{fulllineitems}

\index{read\_file() (in module scibeam.core.tofseries)}

\begin{fulllineitems}
\phantomsection\label{\detokenize{scibeam.core:scibeam.core.tofseries.read_file}}\pysiglinewithargsret{\sphinxcode{\sphinxupquote{scibeam.core.tofseries.}}\sphinxbfcode{\sphinxupquote{read\_file}}}{\emph{file\_path}, \emph{lowerBound=None}, \emph{upperBound=None}, \emph{removeOffset=True}, \emph{cols=2}, \emph{usecols=None}, \emph{offset\_margin\_how='outer'}, \emph{offset\_margin\_size=20}, \emph{skiprows=0}, \emph{sep='\textbackslash{}t'}}{}
Read from sngle file and create an instance of TOFSeries
\begin{quote}\begin{description}
\item[{Parameters}] \leavevmode\begin{itemize}
\item {} 
\sphinxstyleliteralstrong{\sphinxupquote{file\_path}} (\sphinxstyleliteralemphasis{\sphinxupquote{str}}) \textendash{} path to file

\item {} 
\sphinxstyleliteralstrong{\sphinxupquote{lowerBound}} (\sphinxstyleliteralemphasis{\sphinxupquote{int}}\sphinxstyleliteralemphasis{\sphinxupquote{ or }}\sphinxstyleliteralemphasis{\sphinxupquote{float}}) \textendash{} time axis lower boundrary limit for data

\item {} 
\sphinxstyleliteralstrong{\sphinxupquote{upperBound}} (\sphinxstyleliteralemphasis{\sphinxupquote{int}}\sphinxstyleliteralemphasis{\sphinxupquote{ or }}\sphinxstyleliteralemphasis{\sphinxupquote{float}}) \textendash{} time axis upper boundrary limit for data

\item {} 
\sphinxstyleliteralstrong{\sphinxupquote{removeOffset}} (\sphinxstyleliteralemphasis{\sphinxupquote{bool}}) \textendash{} if True (default) remove data offset (set floor to 0 in no-signal region)

\item {} 
\sphinxstyleliteralstrong{\sphinxupquote{cols}} (\sphinxstyleliteralemphasis{\sphinxupquote{int}}) \textendash{} Total number columns in the data file

\item {} 
\sphinxstyleliteralstrong{\sphinxupquote{usecols}} (\sphinxstyleliteralemphasis{\sphinxupquote{int}}) \textendash{} The index of column that will be used out of total number of columns cols

\item {} 
\sphinxstyleliteralstrong{\sphinxupquote{offset\_margin\_how}} (\sphinxstyleliteralemphasis{\sphinxupquote{\{"outer"}}\sphinxstyleliteralemphasis{\sphinxupquote{, }}\sphinxstyleliteralemphasis{\sphinxupquote{"outer left"}}\sphinxstyleliteralemphasis{\sphinxupquote{, }}\sphinxstyleliteralemphasis{\sphinxupquote{"out right"}}\sphinxstyleliteralemphasis{\sphinxupquote{, }}\sphinxstyleliteralemphasis{\sphinxupquote{"inner"}}\sphinxstyleliteralemphasis{\sphinxupquote{, }}\sphinxstyleliteralemphasis{\sphinxupquote{"inner left"}}\sphinxstyleliteralemphasis{\sphinxupquote{, }}\sphinxstyleliteralemphasis{\sphinxupquote{"inner right"\}}}\sphinxstyleliteralemphasis{\sphinxupquote{, }}\sphinxstyleliteralemphasis{\sphinxupquote{default "outer"}}) \textendash{} 
Specify the way to handle offset margin, offset floor value is calculated by averaging the
value in a given range relative to data lower and upper boundrary, with avaliable options:
\begin{itemize}
\item {} 
”outer” (default):  from both left and right side out of the {[}lowerBound, upperBound{]} region

\item {} 
”outer left”: like “outer” but from only left side

\item {} 
”outer right”: like “outer” but from only right side

\item {} 
”inner”: from both left and right side inside of the {[}lowerBound, upperBound{]} region

\item {} 
”inner left”: like “inner” but from only left side

\item {} 
”inner right”: like “inner” but from only left side

\end{itemize}


\item {} 
\sphinxstyleliteralstrong{\sphinxupquote{offset\_margin\_size}} (\sphinxstyleliteralemphasis{\sphinxupquote{int}}) \textendash{} Number of values to use for averaging when calculating offset

\item {} 
\sphinxstyleliteralstrong{\sphinxupquote{skiprows}} (\sphinxstyleliteralemphasis{\sphinxupquote{int}}) \textendash{} number of rows to skip when read in data

\item {} 
\sphinxstyleliteralstrong{\sphinxupquote{sep}} (\sphinxstyleliteralemphasis{\sphinxupquote{str}}\sphinxstyleliteralemphasis{\sphinxupquote{, }}\sphinxstyleliteralemphasis{\sphinxupquote{defult "  "}}) \textendash{} seperator for columns in the data file

\item {} 
\sphinxstyleliteralstrong{\sphinxupquote{Returns}} \textendash{} 

\item {} 
\sphinxstyleliteralstrong{\sphinxupquote{-{-}-{-}-{-}-{-}}} \textendash{} 

\item {} 
\sphinxstyleliteralstrong{\sphinxupquote{of class TOFSeries}} (\sphinxstyleliteralemphasis{\sphinxupquote{Instance}}) \textendash{} 

\end{itemize}

\end{description}\end{quote}

\end{fulllineitems}



\section{Module contents}
\label{\detokenize{scibeam.core:module-scibeam.core}}\label{\detokenize{scibeam.core:module-contents}}\index{scibeam.core (module)}

\chapter{Contribute}
\label{\detokenize{contribute:contribute}}\label{\detokenize{contribute::doc}}
As a open source project, scibeam is under active development towards version 1.0, thus we need contributors from the conmunity.


\section{Steps}
\label{\detokenize{contribute:steps}}\begin{itemize}
\item {} 
Read the \sphinxhref{https://scibeam.readthedocs.io/en/latest/?badge=latest}{documents}

\item {} 
Join the slack channel(\sphinxurl{https://scibeam.slack.com})

\item {} 
Report issure / bug on \sphinxhref{https://github.com/SuperYuLu/SciBeam}{Github}

\item {} 
Look for open \sphinxhref{https://github.com/SuperYuLu/SciBeam/issues}{issues}

\item {} 
Create new pull request

\end{itemize}


\section{Help needed}
\label{\detokenize{contribute:help-needed}}\begin{itemize}
\item {} 
Write unittest for better coverage

\item {} 
Finish document “how to use” part

\item {} 
Add slack channel badge to Readme

\item {} 
Add more file read in format support

\item {} 
Add plotly extension for better visualization

\item {} 
Many more

\end{itemize}


\chapter{scibeam.tests package}
\label{\detokenize{scibeam.tests:scibeam-tests-package}}\label{\detokenize{scibeam.tests::doc}}

\section{Submodules}
\label{\detokenize{scibeam.tests:submodules}}

\section{scibeam.tests.test\_base module}
\label{\detokenize{scibeam.tests:module-scibeam.tests.test_base}}\label{\detokenize{scibeam.tests:scibeam-tests-test-base-module}}\index{scibeam.tests.test\_base (module)}\index{TestFunctions (class in scibeam.tests.test\_base)}

\begin{fulllineitems}
\phantomsection\label{\detokenize{scibeam.tests:scibeam.tests.test_base.TestFunctions}}\pysiglinewithargsret{\sphinxbfcode{\sphinxupquote{class }}\sphinxcode{\sphinxupquote{scibeam.tests.test\_base.}}\sphinxbfcode{\sphinxupquote{TestFunctions}}}{\emph{methodName='runTest'}}{}
Bases: \sphinxcode{\sphinxupquote{unittest.case.TestCase}}
\index{setUp() (scibeam.tests.test\_base.TestFunctions method)}

\begin{fulllineitems}
\phantomsection\label{\detokenize{scibeam.tests:scibeam.tests.test_base.TestFunctions.setUp}}\pysiglinewithargsret{\sphinxbfcode{\sphinxupquote{setUp}}}{}{}
Hook method for setting up the test fixture before exercising it.

\end{fulllineitems}

\index{test\_is\_mixin() (scibeam.tests.test\_base.TestFunctions method)}

\begin{fulllineitems}
\phantomsection\label{\detokenize{scibeam.tests:scibeam.tests.test_base.TestFunctions.test_is_mixin}}\pysiglinewithargsret{\sphinxbfcode{\sphinxupquote{test\_is\_mixin}}}{}{}
\end{fulllineitems}


\end{fulllineitems}



\section{scibeam.tests.test\_common module}
\label{\detokenize{scibeam.tests:module-scibeam.tests.test_common}}\label{\detokenize{scibeam.tests:scibeam-tests-test-common-module}}\index{scibeam.tests.test\_common (module)}\index{TestFunctions (class in scibeam.tests.test\_common)}

\begin{fulllineitems}
\phantomsection\label{\detokenize{scibeam.tests:scibeam.tests.test_common.TestFunctions}}\pysiglinewithargsret{\sphinxbfcode{\sphinxupquote{class }}\sphinxcode{\sphinxupquote{scibeam.tests.test\_common.}}\sphinxbfcode{\sphinxupquote{TestFunctions}}}{\emph{methodName='runTest'}}{}
Bases: \sphinxcode{\sphinxupquote{unittest.case.TestCase}}

Test core.common.py
\index{test\_loadFile() (scibeam.tests.test\_common.TestFunctions method)}

\begin{fulllineitems}
\phantomsection\label{\detokenize{scibeam.tests:scibeam.tests.test_common.TestFunctions.test_loadFile}}\pysiglinewithargsret{\sphinxbfcode{\sphinxupquote{test\_loadFile}}}{}{}
\end{fulllineitems}

\index{test\_winPathHandler() (scibeam.tests.test\_common.TestFunctions method)}

\begin{fulllineitems}
\phantomsection\label{\detokenize{scibeam.tests:scibeam.tests.test_common.TestFunctions.test_winPathHandler}}\pysiglinewithargsret{\sphinxbfcode{\sphinxupquote{test\_winPathHandler}}}{}{}
\end{fulllineitems}


\end{fulllineitems}



\section{scibeam.tests.test\_formatter module}
\label{\detokenize{scibeam.tests:module-scibeam.tests.test_formatter}}\label{\detokenize{scibeam.tests:scibeam-tests-test-formatter-module}}\index{scibeam.tests.test\_formatter (module)}\index{TestFunctions (class in scibeam.tests.test\_formatter)}

\begin{fulllineitems}
\phantomsection\label{\detokenize{scibeam.tests:scibeam.tests.test_formatter.TestFunctions}}\pysiglinewithargsret{\sphinxbfcode{\sphinxupquote{class }}\sphinxcode{\sphinxupquote{scibeam.tests.test\_formatter.}}\sphinxbfcode{\sphinxupquote{TestFunctions}}}{\emph{methodName='runTest'}}{}
Bases: \sphinxcode{\sphinxupquote{unittest.case.TestCase}}
\index{testDict (scibeam.tests.test\_formatter.TestFunctions attribute)}

\begin{fulllineitems}
\phantomsection\label{\detokenize{scibeam.tests:scibeam.tests.test_formatter.TestFunctions.testDict}}\pysigline{\sphinxbfcode{\sphinxupquote{testDict}}\sphinxbfcode{\sphinxupquote{ = \{'a': 1, 'b': 2, 'c': 3.1415926, 'd': 4, 'e': 5\}}}}
\end{fulllineitems}

\index{test\_formart\_dict() (scibeam.tests.test\_formatter.TestFunctions method)}

\begin{fulllineitems}
\phantomsection\label{\detokenize{scibeam.tests:scibeam.tests.test_formatter.TestFunctions.test_formart_dict}}\pysiglinewithargsret{\sphinxbfcode{\sphinxupquote{test\_formart\_dict}}}{}{}
\end{fulllineitems}


\end{fulllineitems}



\section{scibeam.tests.test\_imports module}
\label{\detokenize{scibeam.tests:module-scibeam.tests.test_imports}}\label{\detokenize{scibeam.tests:scibeam-tests-test-imports-module}}\index{scibeam.tests.test\_imports (module)}\index{TestImports (class in scibeam.tests.test\_imports)}

\begin{fulllineitems}
\phantomsection\label{\detokenize{scibeam.tests:scibeam.tests.test_imports.TestImports}}\pysiglinewithargsret{\sphinxbfcode{\sphinxupquote{class }}\sphinxcode{\sphinxupquote{scibeam.tests.test\_imports.}}\sphinxbfcode{\sphinxupquote{TestImports}}}{\emph{methodName='runTest'}}{}
Bases: \sphinxcode{\sphinxupquote{unittest.case.TestCase}}
\index{test\_import\_Gaussian() (scibeam.tests.test\_imports.TestImports method)}

\begin{fulllineitems}
\phantomsection\label{\detokenize{scibeam.tests:scibeam.tests.test_imports.TestImports.test_import_Gaussian}}\pysiglinewithargsret{\sphinxbfcode{\sphinxupquote{test\_import\_Gaussian}}}{}{}
\end{fulllineitems}

\index{test\_import\_PlotTOFSeries() (scibeam.tests.test\_imports.TestImports method)}

\begin{fulllineitems}
\phantomsection\label{\detokenize{scibeam.tests:scibeam.tests.test_imports.TestImports.test_import_PlotTOFSeries}}\pysiglinewithargsret{\sphinxbfcode{\sphinxupquote{test\_import\_PlotTOFSeries}}}{}{}
\end{fulllineitems}

\index{test\_import\_RegMatch() (scibeam.tests.test\_imports.TestImports method)}

\begin{fulllineitems}
\phantomsection\label{\detokenize{scibeam.tests:scibeam.tests.test_imports.TestImports.test_import_RegMatch}}\pysiglinewithargsret{\sphinxbfcode{\sphinxupquote{test\_import\_RegMatch}}}{}{}
\end{fulllineitems}

\index{test\_import\_TOFFrame() (scibeam.tests.test\_imports.TestImports method)}

\begin{fulllineitems}
\phantomsection\label{\detokenize{scibeam.tests:scibeam.tests.test_imports.TestImports.test_import_TOFFrame}}\pysiglinewithargsret{\sphinxbfcode{\sphinxupquote{test\_import\_TOFFrame}}}{}{}
\end{fulllineitems}

\index{test\_import\_TOFSeries() (scibeam.tests.test\_imports.TestImports method)}

\begin{fulllineitems}
\phantomsection\label{\detokenize{scibeam.tests:scibeam.tests.test_imports.TestImports.test_import_TOFSeries}}\pysiglinewithargsret{\sphinxbfcode{\sphinxupquote{test\_import\_TOFSeries}}}{}{}
\end{fulllineitems}

\index{test\_import\_read\_file() (scibeam.tests.test\_imports.TestImports method)}

\begin{fulllineitems}
\phantomsection\label{\detokenize{scibeam.tests:scibeam.tests.test_imports.TestImports.test_import_read_file}}\pysiglinewithargsret{\sphinxbfcode{\sphinxupquote{test\_import\_read\_file}}}{}{}
\end{fulllineitems}

\index{test\_import\_read\_folder() (scibeam.tests.test\_imports.TestImports method)}

\begin{fulllineitems}
\phantomsection\label{\detokenize{scibeam.tests:scibeam.tests.test_imports.TestImports.test_import_read_folder}}\pysiglinewithargsret{\sphinxbfcode{\sphinxupquote{test\_import\_read\_folder}}}{}{}
\end{fulllineitems}


\end{fulllineitems}



\section{scibeam.tests.test\_regexp module}
\label{\detokenize{scibeam.tests:module-scibeam.tests.test_regexp}}\label{\detokenize{scibeam.tests:scibeam-tests-test-regexp-module}}\index{scibeam.tests.test\_regexp (module)}\index{TestRegmatch (class in scibeam.tests.test\_regexp)}

\begin{fulllineitems}
\phantomsection\label{\detokenize{scibeam.tests:scibeam.tests.test_regexp.TestRegmatch}}\pysiglinewithargsret{\sphinxbfcode{\sphinxupquote{class }}\sphinxcode{\sphinxupquote{scibeam.tests.test\_regexp.}}\sphinxbfcode{\sphinxupquote{TestRegmatch}}}{\emph{methodName='runTest'}}{}
Bases: \sphinxcode{\sphinxupquote{unittest.case.TestCase}}
\index{setUp() (scibeam.tests.test\_regexp.TestRegmatch method)}

\begin{fulllineitems}
\phantomsection\label{\detokenize{scibeam.tests:scibeam.tests.test_regexp.TestRegmatch.setUp}}\pysiglinewithargsret{\sphinxbfcode{\sphinxupquote{setUp}}}{}{}
Hook method for setting up the test fixture before exercising it.

\end{fulllineitems}

\index{test\_match() (scibeam.tests.test\_regexp.TestRegmatch method)}

\begin{fulllineitems}
\phantomsection\label{\detokenize{scibeam.tests:scibeam.tests.test_regexp.TestRegmatch.test_match}}\pysiglinewithargsret{\sphinxbfcode{\sphinxupquote{test\_match}}}{}{}
\end{fulllineitems}

\index{test\_matchFolder() (scibeam.tests.test\_regexp.TestRegmatch method)}

\begin{fulllineitems}
\phantomsection\label{\detokenize{scibeam.tests:scibeam.tests.test_regexp.TestRegmatch.test_matchFolder}}\pysiglinewithargsret{\sphinxbfcode{\sphinxupquote{test\_matchFolder}}}{}{}
\end{fulllineitems}

\index{test\_single\_regex\_match() (scibeam.tests.test\_regexp.TestRegmatch method)}

\begin{fulllineitems}
\phantomsection\label{\detokenize{scibeam.tests:scibeam.tests.test_regexp.TestRegmatch.test_single_regex_match}}\pysiglinewithargsret{\sphinxbfcode{\sphinxupquote{test\_single\_regex\_match}}}{}{}
\end{fulllineitems}


\end{fulllineitems}



\section{scibeam.tests.test\_tofseries module}
\label{\detokenize{scibeam.tests:module-scibeam.tests.test_tofseries}}\label{\detokenize{scibeam.tests:scibeam-tests-test-tofseries-module}}\index{scibeam.tests.test\_tofseries (module)}\index{TestFunctions (class in scibeam.tests.test\_tofseries)}

\begin{fulllineitems}
\phantomsection\label{\detokenize{scibeam.tests:scibeam.tests.test_tofseries.TestFunctions}}\pysiglinewithargsret{\sphinxbfcode{\sphinxupquote{class }}\sphinxcode{\sphinxupquote{scibeam.tests.test\_tofseries.}}\sphinxbfcode{\sphinxupquote{TestFunctions}}}{\emph{methodName='runTest'}}{}
Bases: \sphinxcode{\sphinxupquote{unittest.case.TestCase}}
\index{test\_read\_defaults() (scibeam.tests.test\_tofseries.TestFunctions method)}

\begin{fulllineitems}
\phantomsection\label{\detokenize{scibeam.tests:scibeam.tests.test_tofseries.TestFunctions.test_read_defaults}}\pysiglinewithargsret{\sphinxbfcode{\sphinxupquote{test\_read\_defaults}}}{}{}
\end{fulllineitems}

\index{test\_read\_with\_bounds() (scibeam.tests.test\_tofseries.TestFunctions method)}

\begin{fulllineitems}
\phantomsection\label{\detokenize{scibeam.tests:scibeam.tests.test_tofseries.TestFunctions.test_read_with_bounds}}\pysiglinewithargsret{\sphinxbfcode{\sphinxupquote{test\_read\_with\_bounds}}}{}{}
\end{fulllineitems}

\index{test\_read\_without\_offset() (scibeam.tests.test\_tofseries.TestFunctions method)}

\begin{fulllineitems}
\phantomsection\label{\detokenize{scibeam.tests:scibeam.tests.test_tofseries.TestFunctions.test_read_without_offset}}\pysiglinewithargsret{\sphinxbfcode{\sphinxupquote{test\_read\_without\_offset}}}{}{}
\end{fulllineitems}


\end{fulllineitems}

\index{TestTOFSeries (class in scibeam.tests.test\_tofseries)}

\begin{fulllineitems}
\phantomsection\label{\detokenize{scibeam.tests:scibeam.tests.test_tofseries.TestTOFSeries}}\pysiglinewithargsret{\sphinxbfcode{\sphinxupquote{class }}\sphinxcode{\sphinxupquote{scibeam.tests.test\_tofseries.}}\sphinxbfcode{\sphinxupquote{TestTOFSeries}}}{\emph{methodName='runTest'}}{}
Bases: \sphinxcode{\sphinxupquote{unittest.case.TestCase}}
\index{setUp() (scibeam.tests.test\_tofseries.TestTOFSeries method)}

\begin{fulllineitems}
\phantomsection\label{\detokenize{scibeam.tests:scibeam.tests.test_tofseries.TestTOFSeries.setUp}}\pysiglinewithargsret{\sphinxbfcode{\sphinxupquote{setUp}}}{}{}
Hook method for setting up the test fixture before exercising it.

\end{fulllineitems}

\index{test\_find\_time\_idx() (scibeam.tests.test\_tofseries.TestTOFSeries method)}

\begin{fulllineitems}
\phantomsection\label{\detokenize{scibeam.tests:scibeam.tests.test_tofseries.TestTOFSeries.test_find_time_idx}}\pysiglinewithargsret{\sphinxbfcode{\sphinxupquote{test\_find\_time\_idx}}}{}{}
\end{fulllineitems}

\index{test\_init() (scibeam.tests.test\_tofseries.TestTOFSeries method)}

\begin{fulllineitems}
\phantomsection\label{\detokenize{scibeam.tests:scibeam.tests.test_tofseries.TestTOFSeries.test_init}}\pysiglinewithargsret{\sphinxbfcode{\sphinxupquote{test\_init}}}{}{}
\end{fulllineitems}

\index{test\_remove\_data\_offset() (scibeam.tests.test\_tofseries.TestTOFSeries method)}

\begin{fulllineitems}
\phantomsection\label{\detokenize{scibeam.tests:scibeam.tests.test_tofseries.TestTOFSeries.test_remove_data_offset}}\pysiglinewithargsret{\sphinxbfcode{\sphinxupquote{test\_remove\_data\_offset}}}{}{}
\end{fulllineitems}

\index{test\_selectTimeSlice() (scibeam.tests.test\_tofseries.TestTOFSeries method)}

\begin{fulllineitems}
\phantomsection\label{\detokenize{scibeam.tests:scibeam.tests.test_tofseries.TestTOFSeries.test_selectTimeSlice}}\pysiglinewithargsret{\sphinxbfcode{\sphinxupquote{test\_selectTimeSlice}}}{}{}
\end{fulllineitems}


\end{fulllineitems}



\section{Module contents}
\label{\detokenize{scibeam.tests:module-scibeam.tests}}\label{\detokenize{scibeam.tests:module-contents}}\index{scibeam.tests (module)}

\chapter{Indices and tables}
\label{\detokenize{index:indices-and-tables}}\begin{itemize}
\item {} 
\DUrole{xref,std,std-ref}{genindex}

\item {} 
\DUrole{xref,std,std-ref}{modindex}

\item {} 
\DUrole{xref,std,std-ref}{search}

\end{itemize}


\renewcommand{\indexname}{Python Module Index}
\begin{sphinxtheindex}
\def\bigletter#1{{\Large\sffamily#1}\nopagebreak\vspace{1mm}}
\bigletter{s}
\item {\sphinxstyleindexentry{scibeam.core}}\sphinxstyleindexpageref{scibeam.core:\detokenize{module-scibeam.core}}
\item {\sphinxstyleindexentry{scibeam.core.base}}\sphinxstyleindexpageref{scibeam.core:\detokenize{module-scibeam.core.base}}
\item {\sphinxstyleindexentry{scibeam.core.common}}\sphinxstyleindexpageref{scibeam.core:\detokenize{module-scibeam.core.common}}
\item {\sphinxstyleindexentry{scibeam.core.descriptor}}\sphinxstyleindexpageref{scibeam.core:\detokenize{module-scibeam.core.descriptor}}
\item {\sphinxstyleindexentry{scibeam.core.formatter}}\sphinxstyleindexpageref{scibeam.core:\detokenize{module-scibeam.core.formatter}}
\item {\sphinxstyleindexentry{scibeam.core.gaussian}}\sphinxstyleindexpageref{scibeam.core:\detokenize{module-scibeam.core.gaussian}}
\item {\sphinxstyleindexentry{scibeam.core.numerical}}\sphinxstyleindexpageref{scibeam.core:\detokenize{module-scibeam.core.numerical}}
\item {\sphinxstyleindexentry{scibeam.core.peak}}\sphinxstyleindexpageref{scibeam.core:\detokenize{module-scibeam.core.peak}}
\item {\sphinxstyleindexentry{scibeam.core.plot}}\sphinxstyleindexpageref{scibeam.core:\detokenize{module-scibeam.core.plot}}
\item {\sphinxstyleindexentry{scibeam.core.regexp}}\sphinxstyleindexpageref{scibeam.core:\detokenize{module-scibeam.core.regexp}}
\item {\sphinxstyleindexentry{scibeam.core.tofframe}}\sphinxstyleindexpageref{scibeam.core:\detokenize{module-scibeam.core.tofframe}}
\item {\sphinxstyleindexentry{scibeam.core.tofseries}}\sphinxstyleindexpageref{scibeam.core:\detokenize{module-scibeam.core.tofseries}}
\item {\sphinxstyleindexentry{scibeam.tests}}\sphinxstyleindexpageref{scibeam.tests:\detokenize{module-scibeam.tests}}
\item {\sphinxstyleindexentry{scibeam.tests.test\_base}}\sphinxstyleindexpageref{scibeam.tests:\detokenize{module-scibeam.tests.test_base}}
\item {\sphinxstyleindexentry{scibeam.tests.test\_common}}\sphinxstyleindexpageref{scibeam.tests:\detokenize{module-scibeam.tests.test_common}}
\item {\sphinxstyleindexentry{scibeam.tests.test\_formatter}}\sphinxstyleindexpageref{scibeam.tests:\detokenize{module-scibeam.tests.test_formatter}}
\item {\sphinxstyleindexentry{scibeam.tests.test\_imports}}\sphinxstyleindexpageref{scibeam.tests:\detokenize{module-scibeam.tests.test_imports}}
\item {\sphinxstyleindexentry{scibeam.tests.test\_regexp}}\sphinxstyleindexpageref{scibeam.tests:\detokenize{module-scibeam.tests.test_regexp}}
\item {\sphinxstyleindexentry{scibeam.tests.test\_tofseries}}\sphinxstyleindexpageref{scibeam.tests:\detokenize{module-scibeam.tests.test_tofseries}}
\end{sphinxtheindex}

\renewcommand{\indexname}{Index}
\printindex
\end{document}